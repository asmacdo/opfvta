\reviewersection

\begin{point}
	Results, 2nd paragraph --- What is the VTA mean t statistic? (Guess: Is this referring to the fMRI signal in a VTA region of interest?)
\end{point}
\begin{reply}
	We thank the reviewer for pointing out this ambiguity, and have now explained the metric at greater length.
	Indeed the intuition of the reviewer was correct, but we concede that our phrasing was too parsimonious.
\end{reply}

\begin{point}
	Results, 2nd paragraph --- What is stimulation target depth? (Guess: distance from the skull.) I gather that you not only use different stimulation protocols but also different stimulation coordinates. What is the systematics here? What are your experimental parameters?
\end{point}
\begin{reply}
	We thank the reviewer for pointing out this additional ambiguity, and have now detailed the precise meaning of the experimental parameters.
\end{reply}

\begin{point}
	Results, 2nd paragraph --- What is PA? (Guess: Is this anterior-posterior coordinates of the stimulation? Define abbreviation.)
\end{point}
\begin{reply}
	We thank the reviewer for noting this error, indeed the abbreviation was used before its introduction.
	We have now remedied this, and explain the abbreviation --- indeed it stands for posteroanterior --- as well as that it is relative to bregma, when we first introduce it.
\end{reply}

\begin{point}
	Third paragraph --- What is “the entire model, including target coordinates”?
\end{point}
\begin{reply}
	We thank the reviewer for spotting this omission and have now explicitly stated the characteristics of the model.
	In brief for purposes of this reply letter, it refers to the fact that all parameters on which there was recorded variation were included in the statistical model, and variance was distributed among them in an unbiased fashion via a type 3 ANOVA.
\end{reply}

\begin{point}
	Figure 2 --- What are “counts”?
\end{point}
\begin{reply}
	We thank the reviewer for raising the use of this uncommon terminology to our attention.
	We have now updated the name to the more widely used “n”, and noted in the caption that it refers to the sample size.
\end{reply}

\begin{point}
	Why do the authors only stimulate the right VTA (p. 3, first paragraph of Results)?
\end{point}
\begin{reply}
	The rationale was to observe laterality.
	Sadly, on account of viral diffusion and higher than expected light tissue penetration, we can only make tentative statements with regard to laterality based on the current data.
	We have now included a wider discussion on this matter, and how it relates to our suggestions for assay refinement.
\end{reply}

\begin{point}
	Can the authors please also discuss their findings in the light of Brocka et al., Neuroimage 2018, 177:88-97?
\end{point}
\begin{reply}
	We thank the reviewer for highlighting the important contribution which our study can make to supplement the claims made by Brocka et al. --- and have now added a corresponding discussion.
\end{reply}
