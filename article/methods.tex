\section{Methods}

\subsection{Animal Preparation}

To specifically target dopaminergic neurons for optogenetic stimulation, a mouse strain was chosen which expresses Cre recombinase under the Dopamine Transporter (DAT) promoter \cite{Scott2005}.
%!!! update citation
Genotyping via polymerase chain reaction (PCR) using primers listed under !!! was employed to ascertain presence
of the transgenic construct, and a total of \py{pytex_printonly('scripts/animals_tg.py')} transgenic animals and \py{pytex_printonly('scripts/animals_wt.py')} control wildtype animals are included in the study.

The VTA of the animals was injected with a solution containing recombinant Adeno-Associated Viruses (rAAVs).
The vector delivered a plasmid containing a floxed Channelrhodopsin and YFP construct ---
pAAV-EF1a-double floxed-hChR2(H134R)-EYFP-WPRE-HGHpA, a gift from Karl Deisseroth (\href{https://www.addgene.org/20298/}{Addgene plasmid \#20298}).

The measured animals were fitted with an optic fiber implant ($\mathrm{l=\SI{4.7}{\milli\meter} \ d=\SI{400}{\micro\meter} \ NA=0.22}$) targeting the right Ventral Tegmental Area of the midbrain.

The right side of the VTA was stimulated via an Omicron LuxX 488-60 laser (\SI{488}{\nano\meter}) tuned to \SI{30}{\milli\watt} at contact with the fiber implant, according to a series of protocols listed in \cref{tab:stim}.


\subsection{MR Acquisition}

All data are acquired with a Bruker PharmaScan system (\SI{7}{\tesla}, \SI{16}{\centi\meter} bore), and an in-house T/R coil, engineered to permit optic fiber implant protrusion.

Anatomical scans are acquired via a TurboRARE sequence, with a RARE factor of 8, an echo time (TE) of \SI{30}{\milli\second}, an inter-echo spacing of \SI{10}{\milli\second}, and a repetition time (TR) of \SI{2.95}{\second}.
The signal is sampled at a sagittal and horizontal resolution of $\mathrm{\Delta x(\nu)=\Delta y(\phi)=\SI{75}{\micro\meter}}$, and a coronal resolution of $\mathrm{\Delta z(t)=\SI{450}{\micro\meter}}$.

Functional scans are acquired with a flip angle of \SI{60}{\degree} and with $\mathrm{TR/TE = \SI{1000}{\milli\second}/\SI{5.9}{\milli\second}}$.
The signal is sampled at $\mathrm{\Delta x(\nu)=\Delta y(\phi)=\SI{225}{\micro\meter}}$, and $\mathrm{\Delta z(t)=\SI{450}{\micro\meter}}$ .
The functional contrast reflects the cerebral blood volume (CBV) and is generated by
%!!! Endorem dosage
an iron oxide nanoparticle based contrast agent (Endorem, Laboratoire Guebet SA, France).
This contrast is chosen to enable short echo-time imaging and minimize artefacts.


\subsection{Statistics}
We perform statistical modelling with the Statsmodels \cite{statsmodels} package, and incrementally hone in on variance sources, as the qualitative modelling of our variables precludes robust estimation of high-order interaction terms.


\subsection{Reproducibility}

The source code for this document and all data analysis shown herein (including registration and QC workflow execution) is published according to the RepSeP specifications \cite{repsep}.
The data analysis execution and document compilation has been tested repeatedly on numerous machines, and as such we attest that all figures and statistics presented can be reproduced based solely on the raw data, dependency list, and analysis scripts which we distribute.
