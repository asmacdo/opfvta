\section{Methods}

\subsection{Animal Preparation}

To specifically target dopaminergic neurons for optogenetic stimulation, a mouse strain was chosen which expresses Cre recombinase under the Dopamine Transporter (DAT) promoter !!!Besh!!!cite.
Genotyping via polymerase chain reaction (PCR) using primers listed under !!! was employed to ascertain transgenic construct presence, and a total of \sessionpy{pytex_printonly('scripts/animals_tg.py')} transgenic animals and \sessionpy{pytex_printonly('scripts/animals_wt.py')} control wildtype animals are included in the study.

The VTA of the animals was injected with a solution containing recombinant Adeno-Associated Viruses (rAAVs).
The vector delivered a plasmid containing a floxed Channelrhodopsin and YFP construct ---
pAAV-EF1a-double floxed-hChR2(H134R)-EYFP-WPRE-HGHpA, a gift from Karl Deisseroth (\href{https://www.addgene.org/20298/}{Addgene plasmid \#20298}).
Construct expression was ascertained post mortem via fluorescent microscopy of formaldehyde fixated \SI{200}{\milli\metre} brain slices.

The measured animals were fitted with an optic fiber implant ($\mathrm{l=\SI{4.7}{\milli\meter} \ d=\SI{400}{\micro\meter} \ NA=0.22}$) targeting the right Ventral Tegmental Area in the midbrain.
The implant target coordinates were varied, covering the set depicted in \cref{fig:impl_block,fig:impl_phasic}.

The right side of the VTA was stimulated via an Omicron LuxX 488-60 laser (\SI{488}{\nano\meter}) tuned to \SI{30}{\milli\watt} at contact with the fiber implant, according to a series of protocols listed in \cref{tab:stim}.

\subsection{MR Acquisition}

All data were acquired with a Bruker PharmaScan system (\SI{7}{\tesla}, \SI{16}{\centi\meter} bore), and an in-house T/R coil, engineered to permit optic fiber implant protrusion.

Anatomical scans were acquired via a TurboRARE sequence, with a RARE factor of 8, an echo time (TE) of \SI{30}{\milli\second}, an inter-echo spacing of \SI{10}{\milli\second}, and a repetition time (TR) of \SI{2.95}{\second}.
The signal was sampled at a sagittal and horizontal resolution of $\mathrm{\Delta x(\nu)=\Delta y(\phi)=\SI{75}{\micro\meter}}$, and a coronal resolution of $\mathrm{\Delta z(t)=\SI{450}{\micro\meter}}$.

Functional scans were acquired with a flip angle of \SI{60}{\degree} and with $\mathrm{TR/TE = \SI{1000}{\milli\second}/\SI{5.9}{\milli\second}}$.
The signal is sampled at $\mathrm{\Delta x(\nu)=\Delta y(\phi)=\SI{225}{\micro\meter}}$, and $\mathrm{\Delta z(t)=\SI{450}{\micro\meter}}$ .
The functional contrast reflects the cerebral blood volume (CBV) and is generated by an iron oxide nanoparticle based contrast agent (Endorem, Laboratoire Guebet SA, France), delivered as bolus \SI{10}{\minute} before functional scan acquisition to achieve an iron concentration of $\SI{30.24}{\micro\gram\per\gram}$ of body weight.
This contrast is chosen to enable short echo-time imaging and minimize artefacts.

\subsection{Preprocessing}

Data conversion from the proprietary ParaVision format was performed via the Bruker-to-BIDS repositing pipeline \cite{aowsis} of the SAMRI package (version \textcolor{mg}{\texttt{0.4}} \cite{samri}).
Following conversion, data was dummy-scan corrected, registered, and subject to controlled smoothing via the SAMRI registration workflow \cite{irsabi}.
The first 10 volumes were discarded as part of dummy scan correction (automatically including volumes excluded by the scanner software).
Registration was performed using a mouse-brain-optimized parameter set for ANTs (version \textcolor{mg}{\texttt{2.3.1}} \cite{ants}), and data was transformed to a stereotactically oriented standard space (version \textcolor{mg}{\texttt{0.5.3}} \cite{atlases_generator}) based on a high-resolution $\mathrm{T_2}$-weighted atlas \cite{dsu}.
Controlled spatial smoothing was applied in the coronal plane up to \SI{250}{\micro\meter}.

The registered time course data was frequency filtered depending on the analysis workflow.
For stimulus-evoked activity, the data was low-pass filtered at a period threshold of \SI{225}{\second}, and for seed-based functional connectivity, the data was band-pass filtered within a period range of \SIrange{2}{225}{\second}.

\subsection{Statistics}
Volumetric data was modelled using functions from the FSL software package (version \textcolor{mg}{\texttt{5.0.11}}).
First-level regression was applied to the temporally resolved volumetric data via FSL's \textcolor{mg}{\texttt{glm}} function, whereas the second-level analysis was applied to the first-level contrast and variance estimates via FSL's \textcolor{mg}{\texttt{flameo}}.

Stimulus-evoked first-level regression was based on a convolution of the stimulus sequence with an opto-fMRI impulse response function, estimated by a beta fit of the responses reported in \cite{Grandjean2019}.
Seed-based functional connectivity analysis was based regression using the time course of the voxel most sensitive to the stimulus-evoked activity (per scan) in the VTA region of interest.

Higher-level statistical modelling was performed with the Statsmodels (version \textcolor{mg}{\texttt{0.9.9}} \cite{statsmodels}) package.
Model parameters were estimated using the ordinary least squares method, and a type 3 analysis of variance (ANOVA) was employed to control estimate variability for unbalanced categories.

\subsection{Reproducibility}

The source code for this document and all data analysis shown herein (including registration and QC workflow execution) is published according to the RepSeP specifications \cite{repsep}.
The BIDS \cite{bids} data archive which serves as the raw data recourse for this document is openly distributed \cite{opfvta_bidsdata}, as is the full instruction set for recreating this document \cite{me}.
The data analysis execution and document compilation has been tested repeatedly on numerous machines, and as such we attest that all figures and statistics presented can be reproduced based solely on the raw data, dependency list, and analysis scripts which we distribute.
