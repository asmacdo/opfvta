\section{Results}

We note that the VTA mean t statistic is sensitive to the stimulation protocol
(\py{boilerplate.anova(expression='C(Q("Depth rel. skull [mm]")) + C(Q("PA rel. Bregma [mm]")) + Task',factor='Task')}),
the depth
(\py{boilerplate.anova(expression='C(Q("Depth rel. skull [mm]")) + C(Q("PA rel. Bregma [mm]")) + Task',factor='C(Q("Depth rel. skull [mm]"))')})
bot not the PA coordinates
(\py{boilerplate.anova(expression='C(Q("Depth rel. skull [mm]")) + C(Q("PA rel. Bregma [mm]")) + Task',factor='C(Q("PA rel. Bregma [mm]"))')}).

(\py{boilerplate.anova(expression='C(Q("Depth rel. skull [mm]")) + C(Q("PA rel. Bregma [mm]")) + Q("Task Category")',factor='Q("Task Category")')}),

For block stimulation protocols we observe the depth
(\py{boilerplate.anova_block(expression='C(Q("Depth rel. skull [mm]")) + C(Q("PA rel. Bregma [mm]"))',factor='C(Q("Depth rel. skull [mm]"))')})
or the PA coordinates
(\py{boilerplate.anova_block(expression='C(Q("Depth rel. skull [mm]")) + C(Q("PA rel. Bregma [mm]"))',factor='C(Q("PA rel. Bregma [mm]"))')}).

\begin{sansmath}
\py{pytex_subfigs(
        [
                {'script':'scripts/taskgroup.py', 'label':'subjects', 'conf':'article/1col.conf', 'options_pre':'{.48\\textwidth}',
                        'caption':'Task group comparison for animals targeted with all implant coordinates. \\vspace{1em}'
                        },
                {'script':'scripts/implant_coordinates.py', 'label':'subjects', 'conf':'article/1col.conf', 'options_pre':'{.48\\textwidth}',
			'caption':'2D implant coordinate comparison for block stimulation scans only. \\vspace{1em}'
                        },
                ],
        caption='
                Multivariate (protocol and operative feature) comparisons of signal intensity in the VTA region of interest.
                ',
        label='fig:var',
        )}
\end{sansmath}

\begin{sansmath}
\py{pytex_subfigs(
        [
                {'script':'scripts/map_implant_selection.py', 'label':'subjects', 'conf':'article/1col_map.conf', 'options_pre':'{.48\\textwidth}',
                        'caption':'Best coordinate group with slices centered on VTA. \\vspace{1.2em}'
                        },
                {'script':'scripts/map_implant_selection_auto.py', 'label':'subjects', 'conf':'article/1col_map.conf', 'options_pre':'{.48\\textwidth}',
                        'caption':'Best coordinate group centered on langest cluster. \\vspace{1.2em}'
                        },
                {'script':'scripts/map_implant_rest.py', 'label':'subjects', 'conf':'article/1col_map.conf', 'options_pre':'{.48\\textwidth}',
			'caption':'Remaining coordinate group centered on VTA.'
                        },
                {'script':'scripts/map_implant_rest_auto.py', 'label':'subjects', 'conf':'article/1col_map.conf', 'options_pre':'{.48\\textwidth}',
			'caption':'Remaining coordinate group centered on largest cluster.'
                        },
                ],
        caption='
                \\textbf{Best coordinate group scans elicit activity in the Striatum and the Nucleus Accumbens, whereas remaining scans do not.}
                Depicted are statistical maps (thresholded at $\mathrm{\lvert t \\rvert \geq 3}$) of the second-level analysis for block stimulation protocols, comparing different subject groups segmented by implant coordinates --- best coordinate group ($\mathrm{PA \geq -3.3 ; IS \geq -4.4}$) and remaining scans.
                Slices are centered on VTA coordinates ($\mathrm{RAS = 0.5/-3.2/-4.5}$) and on the largest cluster, respectively.
                ',
        label='fig:var',
        )}
\end{sansmath}
