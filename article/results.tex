\section{Results}

We note that the VTA mean t statistic is sensitive to the stimulation protocol category (block and phasic stimulation \py{boilerplate.anova(expression='C(Q("Depth rel. skull [mm]"))*C(Q("PA rel. Bregma [mm]")) + Q("Task Category")',factor='Q("Task Category")')}).

the depth
(\py{boilerplate.anova(expression='C(Q("Depth rel. skull [mm]"))+C(Q("PA rel. Bregma [mm]"))',factor='C(Q("Depth rel. skull [mm]"))')})
bot not the PA coordinates
(\py{boilerplate.anova(expression='C(Q("Depth rel. skull [mm]"))+C(Q("PA rel. Bregma [mm]"))',factor='C(Q("PA rel. Bregma [mm]"))')}).

(\py{boilerplate.anova(expression='C(Q("Depth rel. skull [mm]"))+C(Q("PA rel. Bregma [mm]")) + Q("Task Category")',factor='Q("Task Category")')}),

We note that the VTA mean t statistic is sensitive to the stimulation protocol
(\sessionpy{boilerplate.anova(expression='C(Q("Depth rel. skull [mm]"))*C(Q("PA rel. Bregma [mm]")) + Q("Task Category")',factor='Q("Task Category")')}),
the depth
(\sessionpy{boilerplate.anova(expression='C(Q("Depth rel. skull [mm]"))*C(Q("PA rel. Bregma [mm]")) + Q("Task Category")',factor='C(Q("Depth rel. skull [mm]"))')}),
bot not the PA coordinates
(\sessionpy{boilerplate.anova(expression='C(Q("Depth rel. skull [mm]"))*C(Q("PA rel. Bregma [mm]")) + Q("Task Category")',factor='C(Q("PA rel. Bregma [mm]"))')}),
and the interaction thereof
(\sessionpy{boilerplate.anova(expression='C(Q("Depth rel. skull [mm]"))*C(Q("PA rel. Bregma [mm]")) + Q("Task Category")',factor='C(Q("Depth rel. skull [mm]")):C(Q("PA rel. Bregma [mm]"))')}),
(\sessionpy{boilerplate.anova(expression='C(Q("Depth rel. skull [mm]")):C(Q("PA rel. Bregma [mm]")) + Q("Task Category")',factor='C(Q("Depth rel. skull [mm]")):C(Q("PA rel. Bregma [mm]"))')}),
.

\begin{sansmath}
\sessionpy{pytex_subfigs(
        [
                {'script':'scripts/taskgroup.py', 'conf':'article/3x1.conf', 'options_pre':'{.322\\textwidth}',
                        'caption':'Task group comparison for animals targeted at all explored combinations of implant coordinates.',
                        'label':'mvt',
                        },
                {'script':'scripts/implant_coordinates_block.py', 'conf':'article/3x1_coordinates.conf', 'options_pre':'{.322\\textwidth}',
                        'caption':'Implant coordinate comparison for block stimulation trials (dotted markers indicate best category).',
                        'label':'mvib',
                        },
                {'script':'scripts/implant_coordinates_phasic.py', 'conf':'article/3x1_coordinates.conf', 'options_pre':'{.322\\textwidth}',
                        'caption':'Implant coordinate comparison for phasic stimulation trials (dotted markers indicate best category).',
                        'label':'mvip',
                        },
                ],
        caption='
                \\textbf{We resolve the sensitivity of VTA activation to both the stimulation protocol category and the implant coordinates.}
                Depicted are multivariate (protocol and operative feature) comparisons of signal intensity in the VTA region of interest.
                ',
        label='fig:mv',
        )}
\end{sansmath}

\begin{sansmath}
\sessionpy{pytex_subfigs(
        [
                {'script':'scripts/map_block_filtered_controlled.py', 'label':'subjects', 'conf':'article/2x1_map.conf', 'options_pre':'{.48\\textwidth}',
                        'caption':'Slices centered on VTA. \\vspace{-.2em}',
                        'label':'filtered_map',
                        },
                {'script':'scripts/map_block_filtered_controlled_auto.py', 'label':'subjects', 'conf':'article/2x1_map.conf', 'options_pre':'{.48\\textwidth}',
                        'caption':'Slices centered on langest cluster. \\vspace{-.2em}',
                        'label':'filtered_mapa',
                        },
                {'script':'scripts/distributions_block_filtered_controlled.py', 'label':'tasks', 'conf':'article/distributions.conf', 'options_pre':'{.95\\textwidth}',
                        'caption':'Distribution densities of t-statistic values in the 10 most strongly activated areas.',
                        'label':'filtered_dist',
                        },
                ],
        caption='
                \\textbf{Block stimulation elicits strong and clustered activity in the ventral striatum, particularly the nucleus accumbens.}
                Depicted are the results of the second-level analysis for block stimulus evoked activity observed in the best implant group animals, and corrected for the negative control baseline.
                The figures show volumetric population t-statistic maps \\textbf{(\subref{fig:filtered_map}, \subref{fig:filtered_mapa})} thresholded at $\mathrm{t \geq 3}$, as well as a break-down of activation along atlas parcellation regions \\textbf{(\subref{fig:filtered_dist})}.
                ',
        label='fig:filtered',
        options_pre='\\centering',
        )}
\end{sansmath}

\begin{sansmath}
\sessionpy{pytex_subfigs(
        [
                {'script':'scripts/map_block_other_controlled.py', 'label':'subjects', 'conf':'article/2x1_map.conf', 'options_pre':'{.48\\textwidth}',
                        'caption':'Slices centered on VTA. \\vspace{-.2em}',
                        'label':'other_map',
                        },
                {'script':'scripts/map_block_other_controlled_auto.py', 'label':'subjects', 'conf':'article/2x1_map.conf', 'options_pre':'{.48\\textwidth}',
                        'caption':'Slices centered on langest cluster. \\vspace{-.2em}',
                        'label':'other_mapa',
                        },
                {'script':'scripts/distributions_block_other_controlled.py', 'label':'tasks', 'conf':'article/distributions.conf', 'options_pre':'{.95\\textwidth}',
                        'caption':'Distribution densities of t-statistic values in the 10 most strongly activated areas.',
                        'label':'other_dist',
                        },
                ],
        caption='
                \\textbf{In the rejected implant category group, stimulus evoked activity in response to block stimulation shows a reduced and differently distributed response.}
                Depicted are results of the second-level analysis for block stimulus evoked activity observed in the rejected implant group animals, and corrected for the negative control baseline.
                The figures volumetric population t-statistic maps \\textbf{(\subref{fig:other_map}, \subref{fig:other_mapa})} thresholded at $\mathrm{t \geq 3}$, as well as a break-down of activation along atlas parcellation regions \\textbf{(\subref{fig:other_dist})}.
                ',
        label='fig:other',
        options_pre='\\centering',
        )}
\end{sansmath}

\begin{sansmath}
\sessionpy{pytex_subfigs(
        [
                {'script':'scripts/map_block_filtered_seed.py', 'label':'subjects', 'conf':'article/2x1_map.conf', 'options_pre':'{.48\\textwidth}',
                        'caption':'Slices centered on VTA. \\vspace{-.2em}',
                        'label':'seed_map',
                        },
                {'script':'scripts/map_block_filtered_seed_auto.py', 'label':'subjects', 'conf':'article/2x1_map.conf', 'options_pre':'{.48\\textwidth}',
                        'caption':'Slices centered on langest cluster. \\vspace{-.2em}',
                        'label':'seed_mapa',
                        },
                {'script':'scripts/distributions_block_filtered_seed.py', 'label':'tasks', 'conf':'article/distributions.conf', 'options_pre':'{.95\\textwidth}',
                        'caption':'Distribution densities in the 10 regions with the highest functional connectivity to the VTA seed.',
                        'label':'seed_dist',
                        },
                ],
        caption='
                \\textbf{Seed-based functional connectivity of VTA shows laterality and a primary projection weighting towards the nucleus accumbens.}
                Depicted are volumetric population t-statistic maps \\textbf{(\subref{fig:seed_map}, \subref{fig:seed_mapa})} thresholded at $\mathrm{t \geq 3}$, as well as a break-down of activation along atlas parcellation regions \\textbf{(\subref{fig:seed_dist})}.
                ',
        label='fig:seed',
        options_pre='\\centering',
        )}
\end{sansmath}

\begin{sansmath}
\sessionpy{pytex_subfigs(
        [
                {'script':'scripts/map_block_control.py', 'label':'subjects', 'conf':'article/2x1_map.conf', 'options_pre':'{.48\\textwidth}',
                        'caption':'Slices centered on VTA. \\vspace{-.2em}',
                        'label':'control_map',
                        },
                {'script':'scripts/map_block_control_auto.py', 'label':'subjects', 'conf':'article/2x1_map.conf', 'options_pre':'{.48\\textwidth}',
                        'caption':'Slices centered on langest cluster. \\vspace{-.2em}',
                        'label':'control_mapa',
                        },
                {'script':'scripts/distributions_block_control.py', 'label':'tasks', 'conf':'article/distributions.conf', 'options_pre':'{.95\\textwidth}',
                        'caption':'Distribution densities of t-statistic values in the 10 most strongly activated areas.',
                        'label':'control_dist',
                        },
                ],
        caption='
                \\textbf{Block stimulation in negative control animals produces no large activation clusters, yet scattered activation hints at some visual excitation and heating artefacts.}
                Depicted are volumetric population t-statistic maps \\textbf{(\subref{fig:control_map}, \subref{fig:control_mapa})} --- thresholded at $\mathrm{t \geq 3}$, as well as a break-down of activation along atlas parcellation regions \\textbf{(\subref{fig:control_dist})}.
                ',
        label='fig:control',
        options_pre='\\centering',
        )}
\end{sansmath}
