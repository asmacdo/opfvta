\section{Results}

We note that the VTA mean t statistic is sensitive to the stimulation protocol category
(block and phasic stimulation \py{boilerplate.anova(expression='C(Q("Depth rel. skull [mm]"))*C(Q("PA rel. Bregma [mm]")) + Q("Task Category") + Task',factor='Q("Task Category")')})
more so than to the individual stimulation protocol
(block and phasic stimulation \py{boilerplate.anova(expression='C(Q("Depth rel. skull [mm]"))*C(Q("PA rel. Bregma [mm]")) + Q("Task Category") + Task',factor='Task')})
i.


the depth
(\py{boilerplate.anova(expression='C(Q("Depth rel. skull [mm]"))+C(Q("PA rel. Bregma [mm]")) + Task',factor='C(Q("Depth rel. skull [mm]"))')})
bot not the PA coordinates
(\py{boilerplate.anova(expression='C(Q("Depth rel. skull [mm]"))+C(Q("PA rel. Bregma [mm]")) + Task',factor='C(Q("PA rel. Bregma [mm]"))')}).

(\py{boilerplate.anova(expression='C(Q("Depth rel. skull [mm]"))+C(Q("PA rel. Bregma [mm]")) + Q("Task Category")',factor='Q("Task Category")')}),

We note that the VTA mean t statistic is sensitive to the stimulation protocol
(\sessionpy{boilerplate.anova(expression='C(Q("Depth rel. skull [mm]"))*C(Q("PA rel. Bregma [mm]")) + Q("Task Category")',factor='Q("Task Category")')}),
the depth
(\sessionpy{boilerplate.anova(expression='C(Q("Depth rel. skull [mm]"))*C(Q("PA rel. Bregma [mm]")) + Q("Task Category")',factor='C(Q("Depth rel. skull [mm]"))')}),
bot not the PA coordinates
(\sessionpy{boilerplate.anova(expression='C(Q("Depth rel. skull [mm]"))*C(Q("PA rel. Bregma [mm]")) + Q("Task Category")',factor='C(Q("PA rel. Bregma [mm]"))')}),
and the interaction thereof
(\sessionpy{boilerplate.anova(expression='C(Q("Depth rel. skull [mm]"))*C(Q("PA rel. Bregma [mm]")) + Q("Task Category")',factor='C(Q("Depth rel. skull [mm]")):C(Q("PA rel. Bregma [mm]"))')}),
(\sessionpy{boilerplate.anova(expression='C(Q("Depth rel. skull [mm]")):C(Q("PA rel. Bregma [mm]")) + Q("Task Category")',factor='C(Q("Depth rel. skull [mm]")):C(Q("PA rel. Bregma [mm]"))')}),
.

\begin{sansmath}
\sessionpy{pytex_subfigs(
        [
                {'script':'scripts/taskgroup.py', 'label':'subjects', 'conf':'article/3x1.conf', 'options_pre':'{.325\\textwidth}',
                        'caption':'Task group comparison for animals targeted at all explored combinations of implant coordinates.'
                        },
                {'script':'scripts/implant_coordinates_block.py', 'label':'subjects', 'conf':'article/3x1.conf', 'options_pre':'{.325\\textwidth}',
			'caption':'Implant coordinate comparison by coordinates and VTA activation in block stimulation trials.'
                        },
                {'script':'scripts/implant_coordinates_phasic.py', 'label':'subjects', 'conf':'article/3x1.conf', 'options_pre':'{.325\\textwidth}',
			'caption':'Implant coordinate comparison by coordinates and VTA activation in phasic stimulation trials.'
                        },
                ],
        caption='
                \\textbf{We resolve the sensitivity of VTA activation significance to both the stimulation protocol category and the implant coordinates.}
                Multivariate (protocol and operative feature) comparisons of signal intensity in the VTA region of interest.
                ',
        label='fig:var',
        )}
\end{sansmath}

\begin{sansmath}
\sessionpy{pytex_subfigs(
        [
                {'script':'scripts/map_block_filtered_controlled.py', 'label':'subjects', 'conf':'article/2x3_map.conf', 'options_pre':'{.48\\textwidth}',
                        'caption':'Block stimulation, slices centered on VTA. \\vspace{1.1em}',
                        'label':'sumb',
                        },
                {'script':'scripts/map_block_filtered_controlled_auto.py', 'label':'subjects', 'conf':'article/2x3_map.conf', 'options_pre':'{.48\\textwidth}',
                        'caption':'Block stimulation, slices centered on langest cluster. \\vspace{1.1em}',
                        'label':'sumba',
                        },
                {'script':'scripts/map_phasic_filtered.py', 'label':'subjects', 'conf':'article/2x3_map.conf', 'options_pre':'{.48\\textwidth}',
			'caption':'Phasic stimulation, slices centered on VTA. \\vspace{1.1em}',
                        'label':'sump',
                        },
                {'script':'scripts/map_phasic_filtered_auto.py', 'label':'subjects', 'conf':'article/2x3_map.conf', 'options_pre':'{.48\\textwidth}',
			'caption':'Phasic stimulation, slices centered on largest cluster. \\vspace{1.1em}',
                        'label':'sumpa',
                        },
                {'script':'scripts/map_block_control.py', 'label':'subjects', 'conf':'article/2x3_map.conf', 'options_pre':'{.48\\textwidth}',
			'caption':'Block stimulation of negative control group, slices centered on VTA.',
                        'label':'sumc',
                        },
                {'script':'scripts/map_block_control_auto.py', 'label':'subjects', 'conf':'article/2x3_map.conf', 'options_pre':'{.48\\textwidth}',
			'caption':'Block stimulation of negative control group, slices centered on largest cluster.',
                        'label':'sumca',
                        },
                ],
        caption='
                \\textbf{Block stimulation elicits activity in the target and rostral projection areas, phasic stimulation elicits activity in the target and caudal projection areas, whereas control group stimulation elicits no coherent activation pattern.}
                Depicted are statistical maps (thresholded at $\mathrm{t \geq 3}$) of the second-level analysis grouped by stimulation protocol category and genotpye.
                Right-side figures are centered on VTA coordinates ($\mathrm{RAS = 0.5/-3.2/-4.5}$) and left hand figures on the largest cluster, respectively.
                ',
        label='fig:sum',
        )}
\end{sansmath}
