\section{Discussion}

\subsection{Whole-Brain Dopaminergic Map}

In this article we present the first whole-brain opto-fMRI map of VTA dopaminergic activity in the mouse.
Published as voxelwise reusable data and discussed in terms of regions of interest in the article text, this constitutes an essential resource for model animal study of the dopaminergic system.
The areas identified as functional VTA dopaminergic targets are largely consistent with histological and electrophysiologic literature (as summarized in \cref{fig:ml}).
This highlights the suitability of opto-fMRI for interrogating the mouse dopaminergic system, while providing additional advantages such as longitudinal recording capability and whole-brain coverage.

The predominant VTA projection area identified both in literature and in our study is the nucleus accumbens.
This area is involved in numerous neuropsychological phenomena, and its activation further supports the method's suitability to resolve meaningful brain function and increase the predictability of novel interventions using the mouse model organism.

Throughout most brain regions, we observe a high degree of correspondence between functional activation and structural projection density.
We do, however, also resolve a number of notable differences between opto-fMRI derived projection areas and the structural substrate of the dopaminergic system.
Overall, the contrast between function and structure shows stronger signal and wider coverage for the functional activation pattern, particularly in projection areas.
Notably, the functional map extends into the contralateral ventral striatum, and ipsilateral dorsal striatum.
While the former corresponds to a substantial overextension of the functional map to the contralateral VTA in the midbrain, an extension to the ipsilateral substantia nigra is not as obvious (\cref{fig:ffc}).
The distinction between the two pairs of dopaminergic nuclei and their two pathways --- mesocorticolimbic (originating in the VTA) and nigrostriatal (originating in the substantia nigra) --- forms the basis for dopaminergic modelling.
In conjunction with other recent literature \cite{Lohani2016,Pan2010}, our results contribute to a body of evidence questioning this strict dichotomy, and suggesting that the VTA may also innervate nigrostriatal projection areas to a functionally significant extent.

The amygdala, hippocampus, and the medial prefrontal cortex --- known targets for VTA dopaminergic projections --- do not reveal strong activation in opto-fMRI.
Comparison with published structural projection data, however, indicates that this is due to low fiber bundle density, as these areas also do not show high amounts of structural projections.
Partly negative t-statistic distributions of the contrast between functional activation and structural projection can be observed in areas adjacent to the amygdala and the medial prefrontal cortex (\cref{fig:mff,fig:dffn}), though it should be noted that they are comparable to differences observed in the cerebellar white matter.

We also present a VTA seed-based connectivity analysis which attempts to differentiate primary activation from subsequent signal transmission (according to a simple 1-step signal relay model depicted in \cref{fig:nm}).
Our results indicate that this analysis is capable of resolving projection areas, but is significantly less powerful than stimulus-evoked analysis (\cref{fig:mbfc}).
VTA seed based analysis highlights only a small number of activation clusters and fails to show signal laterality.
This is an interesting outcome, as --- given the superior performance of stimulus-protocol modelled analysis --- it describes two possible features of dopaminergic neurotransmission in the VTA.
The first is that the relay of primary VTA stimulation has higher fidelity than the fMRI measurement of VTA activity itself (i.e. VTA activity is relayed accurately, but outweighed by measurement noise).
The second is that there is a significant threshold to dopaminergic neurotransmission, by which fMRI-measurable baseline activity is predominantly not propagated (i.e. VTA activity is measured accurately, but is relayed in a strongly filtered fashion).

\subsection{Assay Parameters}

In addition to describing the results of a novel assay, this article presents evidence-based suggestions for assay reuse and refinement.
In particular, we detail the effects of stimulus protocol categories and optogenetic targeting coordinates on the performance of the method.

The optic stimulation target coordinate break-down in \cref{fig:mv} shows that, in block stimulation trials, more rostral coordinates elicit stronger VTA activity.
We see that target depth also significantly influences VTA activation, but to a lesser extent than the PA coordinates.
Based on present data we recommend targeting the optic implant at
a posteroanterior distance of \SI{-3.05}{\milli\meter} from bregma,
a left-right distance of \SIrange{0.5}{0.55}{\milli\meter} from the midline,
and a depth of \SI{4.5}{\milli\meter} from the skull surface.
Additional coordinate exploration might be advisable, though further progression towards bregma may lead to direct stimulation of specific efferent fibers rather than the VTA.

Comparison of implant-coordinate resolved VTA activity between phasic and block stimulation categories indicates that there may be a difference in susceptibility to fiber tip coordinates for the two stimulation types.
Yet, the absence of coherent activity patterns elicited by phasic stimulation (\cref{fig:mvt,fig:phasica}) prevents this from being a firm statement.
The overall low susceptibility to phasic stimulation is most likely due to the intrinsically lower statistical power of such stimulation protocols in fMRI, but might also reflect lower neuronal susceptibility to phasic stimulation.
While this hypothesis is strongly speculative, we highlight that it is testable, and more caudal coordinates should be revisited upon strong alterations to the stimulation protocol.

Regarding the distribution of activation across projection areas, we note a strong and unexpected divergence between the most sensitive (“best”) and least sensitive (“rejected”) implant coordinate category responses to block stimulation (\cref{fig:mbfc,fig:mboc}).
In addition to a difference in VTA and efferent signal intensity (expected as per the selection criterion), we also notice a different pattern of target areas.
Interestingly, the activity pattern elicited in the “rejected” group is more strongly weighted towards the hindbrain.
This effect might be related to the activation of descending dopaminergic projections, though further investigation is needed to clarify this point and, in general, to better understand the cross-connectivity between deep brain nuclei.

The activity patterns in wild type control animals are narrow in their extent (\cref{fig:control}), and --- when controlled for --- do not meaningfully impact the dopaminergic block stimulation contrast (\cref{fig:mbfc,fig:uncontrolled}).
Based on the activation distribution, however, it may be inferred that trace heating artefacts (midbrain activation) and visual stimulation (lateral geniculate nucleus thalamic activation) are present.
On account of this, for further experiments, we suggest using eye occlusion, dark or dark-painted ferrule sleeves, as well as laser intensities lower than the \SI{30}{\milli\watt} (\SI{239}{\milli\watt\per\milli\meter\squared}) used in this study.

While stimulus-evoked activation displayed significant laterality, large clusters of significant activity were also observed on the contralateral side.
Yet, contralateral projection is not a feature representative of the ascending dopaminergic system, and neither structural VTA projection data (\cref{fig:ffc}), nor previous opto-fMRI work in rats \cite{Lohani2016} has indicated such an effect.
Fluorescence microscopy (\cref{fig:ffc}) revealed that the expression of the viral construct injected at the site of the right VTA extends over a large area, that even encompasses the contralateral VTA.
This explains the occurrence of contralateral fMRI responses, which are most likely weaker due to a lower photon fluence at the site of the left VTA.
Together, these data suggest that the virus amount injected for the assay could be significantly reduced, to less than the
\sessionpy{boilerplate.report_bound('VirusInjectionProtocol_amount', 'min', '\micro\litre')}
(\SI{5.7e12}{\vg\per\milli\litre})
used as the minimal volume in this study.

Lastly, the most salient qualitative feature of \cref{fig:h} is the displacement of labelled neurons from the area in the proximity of the optic fiber implant tip.
This feature was consistent across animals and implantation sites, and is cause for concern, since it limits the extent to which cells can be optically stimulated.
Particularly, such a feature could exacerbate limitations arising from heating artefacts, since the maximum SNR attainable via stronger light stimulation may be capped to an unnecessarily low level.
This issue might be mitigated by using thinner optic fiber implants (e.g. $\diameter$\SI{200}{\micro\meter}, as opposed to the $\diameter$\SI{400}{\micro\meter} fibers used in this study).

\subsection{Conclusion}

In this article we present the first whole-brain map of VTA dopaminergic signalling in the mouse, and publish it in a standard space aligned with stereotactic manipulation coordinates \cite{me}.
We determine that the mapping is consistent with known structural projections, and note the instances where differences are observed.
Further, we explore network structure aware analysis via functional connectivity, finding that the assay provides limited support for signal relay imaging.
As part of this work we perform an in-depth investigation of experimental variation, and summarize evidence-based instructions for assay reuse.
The results herein published provide a reference dopaminergic stimulus-evoked functional neurophenotype map and a novel and thoroughly documented workflow for the preclinical imaging of dopaminergic function, both of which are crucial to elucidating the etiology of numerous disorders and improving psychopharmacological interventions in health and disease.
