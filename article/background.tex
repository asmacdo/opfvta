\section{Background}

%The ultimate assessment standard for knowledge is its ability to inform action.
%As such, neuroscience incrementally aims to support the development of meaningful and predictable control of the brain.
%Within this endeavour, it is the most easily accessible components of the brain, given current technology, which can provide the greatest opportunities for advancement.

The dopaminergic system consists of a strongly localized, and widely projecting set of neurons with cell bodies clustered in the midbrain into two lateralized nucleus pairs, the Substantia Nigra pars compacta (SNc) and the Ventral Tegmental Area (VTA, \cref{fig:ml}).
On account of the small number of dopaminergic neurons ($\approx300,000$ in humans \cite{rice2016}, $\approx10,000$ in rats \cite{german1993}, and $\approx4,000$ in mice \cite{triarhou1988}), tractography commonly fails to resolve the degree centrality of this neurotransmitter system, precluding it from being a prominent node in such graph representations of the brain.
However, it is precisely the small number of widely branching and similar neurons, which makes the dopaminergic system a credible candidate for truly node-like function in coordinating brain activity.
As is expected given such salient features, the system is widely implicated in neuropsychiatric phenomena (including
addiction \cite{DiChiara1988,DiChiara1999},
attentional control \cite{Nieoullon2002},
motivation \cite{Salamone1994},
creativity \cite{Chermahini2010},
personality \cite{Depue1999},
neurodegeneration \cite{Masliah2000},
and schizophrenia \cite{Howes2009}),
and is a common target for pharmacological interventions.
Lastly, due to high evolutionary conservation \cite{Yamamoto2011}, the dopaminergic system is also an excellent candidate for translational study.
%Manipulation of the dopaminergic system extends widely beyond the medical field, and includes performance-enhancement \cite{Mehta2000,Turner2003}, as well recreation \cite{DiChiara1988}.
%As such, better and more predictively powerful models of the dopaminergic system can improve numerous aspects of human activity, in the clinic and beyond.


%Experimentation in human subjects is methodologically constrained, and thus neuroscience relies on other context, such as model animal research, for cell biological insights, which are instrumental to the refinement of interventions.
%Of the common model animals, the mouse offers significant advantages including short generation spans, broad availability of transgenic lines, highly accessible histology, and the smallest size among mammalian model organisms.
%Far from trivial, this latter quality offers a significant advantage for numerous imaging techniques, including optics, optoacoustics, and magnetic resonance imaging (MRI).
%Furthermore, the small size greatly increases experiment scalability and multi-center reproducibility, beyond even what can be achieved using the rat model.
%Of particular relevance to pharmacological research, the mouse possesses a high metabolic rate, allowing for rapid drug clearance and thus high contrast in repeated drug administration sessions.
%
%As it coordinates phenomena such as perception, decision, and action, the function of the brain is highly holistic, and this underscores the importance of whole-brain measurement techniques for its characterization.
%This is particularly true for monoaminergic neurotransmitter systems, where focal interventions can have wide-ranging effects.
%Owing to its deep penetration, high rostrocaudal coverage, and large-scale usage in human studies, fMRI is one of the foremost methods for studying dopaminergic modulation of brain function.

Imaging a neurotransmitter system comprised of a small number of cells based only on spontaneous activity is highly unreliable due to an intrinsically low signal to noise ratio (SNR).
This limitation can, however, be overcome by introducing exogenous stimulation.
While the colocalization of widely projecting dopaminergic cell bodies into nuclei renders temporally precise and population-wide targeting feasible, dopaminergic nuclei also contain notable sub-populations of non-dopaminergic cells, which may confound an intended dopaminergic read-out \cite{Taylor2014}.
In order to specifically target dopaminergic cells, they need to be sensitized to an otherwise inert stimulus in a transcription-dependent manner.
This can be achieved via optogenetics, which is based on light-stimulation of cells expressing light-sensitive proteins such as channelrhodopsin \cite{Boyden2005}.
Cell-type selectivity can be achieved by Cre-conditional channelrhodopsin vector delivery \cite{Orban1992} to transgenic animals expressing Cre-recombinase under a dopaminergic promoter.
Following protein expression, stimuli can be delivered via an implanted optic fiber.
The combination of this stimulation method with fMRI is commonly referred to as opto-fMRI and can provide information on functional connectivity between a primary activation site and associated projection areas \cite{Desai2011,Grandjean2019}.

%The majority of dopaminergic cell bodies are clustered in the midbrain into two lateralized nucleus pairs, the Ventral Tegmental Area (VTA) and Substantia Nigra pars compacta (SNc).
%Of these, the VTA displays a wider distribution of efferents (\cref{fig:ml}), whereas the SNc projects primarily to the dorsal striatum \cite{Pan2010}.

%The most popular means of producing spatially resolved sensitivity summaries from fMRI, and opto-fMRI in particular, is general linear modelling (GLM) of stimulus evoked activity \cite{Friston1995}.
%For this purpose, a precisely specified stimulus train is presented to the brain, and convolved with an impulse response function (IRF, also called hemodynamic response function) for analysis.
%Subsequently, a mass univariate regression analysis is performed across all voxels, allowing them to be assigned parameter values, which denote the strength of activation adjusted for the sample noise.
%While powerful, this approach implies that all voxels in the brain are exposed in an indistinguishable fashion to the stimulation, a questionable assumption in view of the actual network architecture.
%Particularly in cases where stimulation is introduced via well-known pathways, a simple network model for signal transmission could be explored via seed-based functional connectivity (\cref{fig:nm}).
%Moreover, for neurotransmitter systems with colocalized cell bodies and long efferent projections, the macroscopic resolution of fMRI allows mapping somatodendritic processes (e.g. excitability upon depolarization at the soma) to “cell-body“ voxels, and signal transmission at the synapse to “projection“ voxels (\cref{fig:cm}).

Key questions surrounding VTA function in preclinical models are, firstly, method feasibility in animal models more accessible to transgenic techniques, such as the mouse; and secondly, a mapping of the efferent spectrum for dopaminergic VTA output.
In particular, in the study of the Rat VTA, it has both been suggested that the efferent dopaminergic spectrum encompasses but extends beyond well-documented structural projections \cite{Lohani2016} --- or alternatively, that VTA dopaminergic efferences are comparatively sparse and that based on translational insight the dopaminergic paradigm of motivation-related VTA function could be questioned \cite{Brocka2018}.

The current study of whole-brain VTA dopaminergic function in mice aims to produce three novel research outputs.
Firstly, a proof-of-principle documenting the feasibility of midbrain dopaminergic opto-fMRI in the mouse should be demonstrated, using a protocol which affords qualitative comparability with extant rat data, such as block stimulation and right VTA targeting.
Pursuing open questions in the field, results should be quantitatively benchmarked with respect to histologically documented structural projections in the mouse.
Secondly, the procedure needs to be optimized by systematic variation of experimental parameters (such as targeting and stimulation protocol variations) in order to ascertain reliability and reproducibility, as is required for a general-purpose dopaminergic system assay.
Lastly, a reference neurophenotype of stimulus-evoked dopaminergic function (represented as a brain-wide voxelwise map) should be published in standard space to facilitate co-registered data integration, operative targeting, and comparative evaluation of pathology or treatment induced effects.

These goals presuppose not only the production of experimental data, but also the development of a transparent, reliable, and publicly accessible analysis workflow, which leverages pre-existing standards for mouse brain data processing \cite{irsabi} and extends them to the statistical analysis.
%Together with such tools and an openly accessible reference experimental dataset, a mouse brain dopaminergic imaging protocol can be reimplemented and derived throughout preclinical neuroimaging applications.
