\renewcommand{\thetable}{S\arabic{table}}
\setcounter{figure}{0}
\renewcommand{\thefigure}{S\arabic{figure}}

\section{Supplementary Materials}
\sessionpy{
        pytex_tab(inner_tabular=boilerplate.events_tab('data/CogB.tsv'),
                label='CogB',
                caption='CogB stimulation protocol.',
                options_pre='[h!] \\scriptsize \\centering \\resizebox{\\columnwidth}{!}{',
                data='data/CogB.tsv',
                options_post='}',
                )
        }
\sessionpy{
        pytex_tab(inner_tabular=boilerplate.events_tab('data/CogBr.tsv'),
                label='CogBr',
                caption='CogBr stimulation protocol.',
                options_pre='[h!] \\scriptsize \\centering \\resizebox{\\columnwidth}{!}{',
                data='data/CogBr.tsv',
                options_post='}',
                )
        }
\sessionpy{
        pytex_tab(inner_tabular=boilerplate.events_tab('data/CogBl.tsv'),
                label='CogBl',
                caption='CogBl stimulation protocol.',
                options_pre='[h!] \\scriptsize \\centering \\resizebox{\\columnwidth}{!}{',
                data='data/CogBl.tsv',
                options_post='}',
                )
        }
\sessionpy{
        pytex_tab(inner_tabular=boilerplate.events_tab('data/CogBm.tsv'),
                label='CogBm',
                caption='CogBm stimulation protocol.',
                options_pre='[h!] \\scriptsize \\centering \\resizebox{\\columnwidth}{!}{',
                data='data/CogBm.tsv',
                options_post='}',
                )
        }
\sessionpy{
        pytex_tab(inner_tabular=boilerplate.events_tab('data/CogMwf.tsv'),
                label='CogMwf',
                caption='CogMwf stimulation protocol.',
                options_pre='[h!] \\scriptsize \\centering \\resizebox{\\columnwidth}{!}{',
                data='data/CogMwf.tsv',
                options_post='}',
                )
        }
\sessionpy{
        pytex_tab(inner_tabular=boilerplate.events_tab('data/CogP.tsv'),
                label='CogP',
                caption='CogP stimulation protocol.',
                options_pre='[h!] \\scriptsize \\centering \\resizebox{\\columnwidth}{!}{',
                data='data/CogP.tsv',
                options_post='}',
                )
        }
\sessionpy{
        pytex_tab(inner_tabular=boilerplate.events_tab('data/JPogP.tsv'),
                label='JPogP',
                caption='JPogP stimulation protocol.',
                options_pre='[h!] \\scriptsize \\centering \\resizebox{\\columnwidth}{!}{',
                data='data/JPogP.tsv',
                options_post='}',
                )
        }


\begin{sansmath}
\py[var]{pytex_subfigs(
        [
                {'script':'scripts/f_vta_subjects_t.py', 'label':'subjects', 'conf':'article/3x1.conf', 'options_pre':'{.35\\textwidth}',
                        },
                {'script':'scripts/implant_depth.py', 'label':'subjects', 'conf':'article/3x1.conf', 'options_pre':'{.32\\textwidth}',
                        },
                {'script':'scripts/implant_pa.py', 'label':'subjects', 'conf':'article/3x1.conf', 'options_pre':'{.32\\textwidth}',
                        },
                ],
        caption='
                Multivariate (depth and posteroantior) implant coordinate comparisons of signal intensity in the VTA region of interest.
                ',
        label='fig:var',
        )}
\end{sansmath}

Modelling interaction effects we observe the depth
(\sessionpy{boilerplate.anova(expression='Q("Task Category")*C(Q("Depth rel. skull [mm]")) + Q("Task Category")*C(Q("PA rel. Bregma [mm]"))',factor='C(Q("Depth rel. skull [mm]"))')})
or the PA coordinates
(\sessionpy{boilerplate.anova(expression='C(Q("Depth rel. skull [mm]")) + C(Q("PA rel. Bregma [mm]"))',factor='C(Q("PA rel. Bregma [mm]"))')}).

For block stimulation protocols we observe the depth
(\sessionpy{boilerplate.anova(expression='C(Q("Depth rel. skull [mm]")) + C(Q("PA rel. Bregma [mm]"))',factor='C(Q("Depth rel. skull [mm]"))', task_category='Block')})
or the PA coordinates
(\sessionpy{boilerplate.anova(expression='C(Q("Depth rel. skull [mm]")) + C(Q("PA rel. Bregma [mm]"))',factor='C(Q("PA rel. Bregma [mm]"))', task_category='Block')}).

\begin{sansmath}
\sessionpy{pytex_subfigs(
        [
                {'script':'scripts/map_block_filtered_controlled_auto_.py', 'conf':'article/3x1_map.conf', 'options_pre':'{.32\\textwidth}',
                        'caption':'Block stimulation of best implant category group.',
                        'label':'bs_f',
                        },
                {'script':'scripts/map_block_other_controlled_auto_.py', 'conf':'article/3x1_map.conf', 'options_pre':'{.32\\textwidth}',
                        'caption':'Block stimulation of rejected implant category group.',
                        'label':'bs_o',
                        },
                {'script':'scripts/implant_coordinates.py', 'conf':'article/3x1_withmap.conf', 'options_pre':'{.32\\textwidth}',
                        'caption':'Implant group segmentation (dotted markers indicate best category).',
                        'label':'bs_c',
                        },
                ],
        caption='
                \\textbf{Implant coordinate categorization based on both block and phasic stimulus trials does not show a better projection segmentation than based on block trials only.}
                Depicted are t-statistic maps (thresholded at $\mathrm{t \geq 3}$) of the second-level analysis for block stimulation protocols, divided into best and rejected category groups \\textbf{(\subref{fig:bs_f}, \subref{fig:bs_o})}.
                Slices are centered on the largest cluster.
                ',
        label='fig:bs_',
        )}
\end{sansmath}

\begin{sansmath}
\sessionpy{pytex_subfigs(
        [
                {'script':'scripts/map_block_filteredManual.py', 'label':'subjects', 'conf':'article/2x1_map.conf', 'options_pre':'{.48\\textwidth}',
                        'caption':'Block stimulation of rostralmost implant category group, slices centered on VTA. \\vspace{1.2em}'
                        },
                {'script':'scripts/map_block_filtered_autoManual.py', 'label':'subjects', 'conf':'article/2x1_map.conf', 'options_pre':'{.48\\textwidth}',
                        'caption':'Block stimulation of rostralmost implant category group, slices centered on largest cluster. \\vspace{1.2em}'
                        },
                {'script':'scripts/map_block_otherManual.py', 'label':'subjects', 'conf':'article/2x1_map.conf', 'options_pre':'{.48\\textwidth}',
                        'caption':'Block stimulation of caudalmost implant category group, slices centered on VTA.'
                        },
                {'script':'scripts/map_block_other_autoManual.py', 'label':'subjects', 'conf':'article/2x1_map.conf', 'options_pre':'{.48\\textwidth}',
                        'caption':'Block stimulation of caudalmost implant category group, slices centered on largest cluster.'
                        },
                ],
        caption='
                \\textbf{Implant coordinate categorization based on a PA coordinate separation at -3.25 does not show better projection segmentation than block stimulus trial categorization.}
                Depicted are statistical maps (thresholded at $\mathrm{t \geq 3}$) of the second-level analysis for block stimulation protocols, divided by implant category group.
                Slices are centered on the VTA coordinates ($\mathrm{RAS = 0.5/-3.2/-4.5}$) and the largest cluster, respectively.
                ',
        label='fig:var',
        )}
\end{sansmath}

\begin{sansmath}
\sessionpy{pytex_subfigs(
        [
                {'script':'scripts/map_block.py', 'label':'subjects', 'conf':'article/2x1_map.conf', 'options_pre':'{.48\\textwidth}',
                        'caption':'Block stimulation, slices centered on VTA. Thresholded at \\vspace{1.2em}'
                        },
                {'script':'scripts/map_block_auto.py', 'label':'subjects', 'conf':'article/2x1_map.conf', 'options_pre':'{.48\\textwidth}',
                        'caption':'Block stimulation, slices centered on langest cluster. \\vspace{1.2em}'
                        },
                {'script':'scripts/map_phasic.py', 'label':'subjects', 'conf':'article/2x1_map.conf', 'options_pre':'{.48\\textwidth}',
                        'caption':'Phasic stimulation, slices centered on VTA.'
                        },
                {'script':'scripts/map_phasic_auto.py', 'label':'subjects', 'conf':'article/2x1_map.conf', 'options_pre':'{.48\\textwidth}',
                        'caption':'Phasic stimulation, slices centered on langest cluster.'
                        },
                ],
        caption='
                \\textbf{No negative activation patterns are salient upon block VTA stimulation, and no coherent activation patterns of any sort after phasic VTA stimulation.}
                Depicted are t-statistic maps (thresholded at $\mathrm{\lvert t \\rvert \geq 3}$) of second-level analyses, divided by stimulation category and binning all implant coordinates.
                Slices are centered on the VTA coordinates ($\mathrm{RAS = 0.5/-3.2/-4.5}$) and the largest cluster, respectively.
                ',
        label='fig:all',
        )}
\end{sansmath}

\begin{sansmath}
\sessionpy{pytex_subfigs(
        [
                {'script':'scripts/map_block_filtered_seed_controlled.py', 'label':'subjects', 'conf':'article/2x1_map.conf', 'options_pre':'{.48\\textwidth}',
                        'caption':'Slices centered on VTA.'
                        },
                {'script':'scripts/map_block_filtered_seed_controlled_auto.py', 'label':'subjects', 'conf':'article/2x1_map.conf', 'options_pre':'{.48\\textwidth}',
                        'caption':'Slices centered on langest cluster.'
                        },
                ],
        caption='
                Depicted are t-statistic maps (thresholded at $\mathrm{t \geq 3}$) of the second-level analysis for block stimulation task VTA seed functional connectivity, observed in the best implant category, corrected for the negative control baseline.
                Slices are centered on the VTA coordinates ($\mathrm{RAS = 0.5/-3.2/-4.5}$) and the largest cluster, respectively.
                This comparison is only provided for the sake of completeness and analogy with the stimulus-evoked analysis.
                Conceptually this comparison is not of primary interest, since seed-based functional connectivity attempts to include precisely the baseline functioning of the system into the evaluation.
                ',
        label='fig:sc',
        )}
\end{sansmath}

\begin{sansmath}
\sessionpy{pytex_subfigs(
        [
                {'script':'scripts/map_block_filtered_auto.py', 'label':'subjects', 'conf':'article/2x1_map.conf', 'options_pre':'{.48\\textwidth}',
                        'caption':'Block stimulation of best implant category group',
                        },
                {'script':'scripts/map_block_other_auto.py', 'label':'subjects', 'conf':'article/2x1_map.conf', 'options_pre':'{.48\\textwidth}',
                        'caption':'Block stimulation of rejected implant category group.',
                        },
                ],
        caption='
                \\textbf{The uncorrected population-level response to block stimulation does not significantly differ from the negative-control corrected results in \cref{fig:filtered_map,fig:other_mapa}.}
                Depicted are t-statistic maps (thresholded at $\mathrm{t \geq 3}$) of the second-level analysis for block stimulation protocols, divided by implant category group.
                Slices are centered on the largest cluster.
                ',
        label='fig:ucontrolled',
        )}
\end{sansmath}
