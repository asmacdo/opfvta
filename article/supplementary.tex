\section{Supplementary Materials}

\begin{sansmath}
\py{pytex_subfigs(
        [
                {'script':'scripts/map_implant_selection_all.py', 'label':'subjects', 'conf':'article/1col_map.conf', 'options_pre':'{.48\\textwidth}',
                        'caption':'Best coordinate group with slices centered on VTA. \\vspace{1.2em}'
                        },
                {'script':'scripts/map_implant_selection_auto_all.py', 'label':'subjects', 'conf':'article/1col_map.conf', 'options_pre':'{.48\\textwidth}',
                        'caption':'Best coordinate group centered on langest cluster. \\vspace{1.2em}'
                        },
                {'script':'scripts/map_implant_rest_all.py', 'label':'subjects', 'conf':'article/1col_map.conf', 'options_pre':'{.48\\textwidth}',
			'caption':'Remaining coordinate group centered on VTA.'
                        },
                {'script':'scripts/map_implant_rest_auto_all.py', 'label':'subjects', 'conf':'article/1col_map.conf', 'options_pre':'{.48\\textwidth}',
			'caption':'Remaining coordinate group centered on largest cluster.'
                        },
                ],
        caption='
                \\textbf{Best coordinate group scans elicit activity in the Striatum and the Nucleus Accumbens, whereas remaining scans do not.}
                Depicted are statistical maps (thresholded at $\mathrm{\lvert t \\rvert \geq 3}$) of the second-level analysis for block stimulation protocols, comparing different subject groups segmented by implant coordinates --- best coordinate group ($\mathrm{PA \geq -3.3 ; IS \geq -4.4}$) and remaining scans.
                Slices are centered on VTA coordinates ($\mathrm{RAS = 0.5 -3.2 -4.5}$) and on the largest cluster, respectively.
                ',
        label='fig:var',
        )}
\end{sansmath}

\begin{sansmath}
\py{pytex_subfigs(
        [
                {'script':'scripts/f_vta_subjects_t.py', 'label':'subjects', 'conf':'article/1col_2x3.conf', 'options_pre':'{.48\\textwidth}',
			'caption':'VTA signal intensity across subjects.'
                        ,},
                {'script':'scripts/f_vta_tasks_t.py', 'label':'tasks','conf':'article/1col_2x3.conf', 'options_pre':'{.48\\textwidth}',
                        'caption':'VTA signal intensity across stimulation protocols.'
                        ,},
                ],
        caption='
                Multivariate (subject and stimulation protocol) comparisons of significance and signal intensity at the whole-brain level or restricted to the VTA region of interest.
                ',
        label='fig:var',
        )}
\end{sansmath}

