\renewcommand{\thetable}{S\arabic{table}}
\setcounter{figure}{0}
\renewcommand{\thefigure}{S\arabic{figure}}

\section{Supplementary Materials}
\sessionpy{
        pytex_tab(inner_tabular=boilerplate.events_tab('data/CogB.tsv'),
                label='CogB',
                caption='“CogB” stimulation protocol.',
                options_pre='[h!] \\scriptsize \\centering \\resizebox{\\columnwidth}{!}{',
                data='data/CogB.tsv',
                options_post='}',
                )
        }
\sessionpy{
        pytex_tab(inner_tabular=boilerplate.events_tab('data/CogBr.tsv'),
                label='CogBr',
                caption='“CogBr” stimulation protocol.',
                options_pre='[h!] \\scriptsize \\centering \\resizebox{\\columnwidth}{!}{',
                data='data/CogBr.tsv',
                options_post='}',
                )
        }
\sessionpy{
        pytex_tab(inner_tabular=boilerplate.events_tab('data/CogBl.tsv'),
                label='CogBl',
                caption='“CogBl” stimulation protocol.',
                options_pre='[h!] \\scriptsize \\centering \\resizebox{\\columnwidth}{!}{',
                data='data/CogBl.tsv',
                options_post='}',
                )
        }
\sessionpy{
        pytex_tab(inner_tabular=boilerplate.events_tab('data/CogBm.tsv'),
                label='CogBm',
                caption='“CogBm” stimulation protocol.',
                options_pre='[h!] \\scriptsize \\centering \\resizebox{\\columnwidth}{!}{',
                data='data/CogBm.tsv',
                options_post='}',
                )
        }
\sessionpy{
        pytex_tab(inner_tabular=boilerplate.events_tab('data/CogMwf.tsv'),
                label='CogMwf',
                caption='“CogMwf” stimulation protocol.',
                options_pre='[h!] \\scriptsize \\centering \\resizebox{\\columnwidth}{!}{',
                data='data/CogMwf.tsv',
                options_post='}',
                )
        }
\sessionpy{
        pytex_tab(inner_tabular=boilerplate.events_tab('data/CogP.tsv'),
                label='CogP',
                caption='“CogP” stimulation protocol.',
                options_pre='[h!] \\scriptsize \\centering \\resizebox{\\columnwidth}{!}{',
                data='data/CogP.tsv',
                options_post='}',
                )
        }
\sessionpy{
        pytex_tab(inner_tabular=boilerplate.events_tab('data/JPogP.tsv'),
                label='JPogP',
                caption='“JPogP” stimulation protocol.',
                options_pre='[h!] \\scriptsize \\centering \\resizebox{\\columnwidth}{!}{',
                data='data/JPogP.tsv',
                options_post='}',
                )
        }


\begin{sansmath}
\sessionpy{pytex_subfigs(
        [
                {'script':'scripts/map_block_filtered_all.py', 'label':'subjects', 'conf':'article/2x1_map.conf', 'options_pre':'{.48\\textwidth}',
                        'caption':'Block stimulation of best implant category group, slices centered on VTA. \\vspace{1.2em}'
                        },
                {'script':'scripts/map_block_filtered_all_auto.py', 'label':'subjects', 'conf':'article/2x1_map.conf', 'options_pre':'{.48\\textwidth}',
                        'caption':'Block stimulation of best implant category group, slices centered on langest cluster. \\vspace{1.2em}'
                        },
                {'script':'scripts/map_block_other_all.py', 'label':'subjects', 'conf':'article/2x1_map.conf', 'options_pre':'{.48\\textwidth}',
                        'caption':'Block stimulation of eliminated implant category group, slices centered on VTA.'
                        },
                {'script':'scripts/map_block_other_all_auto.py', 'label':'subjects', 'conf':'article/2x1_map.conf', 'options_pre':'{.48\\textwidth}',
                        'caption':'Block stimulation of eliminated implant category group, slices centered on langest cluster.'
                        },
                ],
        caption='
                \\textbf{No implant category group elicits coherent negative activation clusters upon block stimulation.}
                Depicted are statistical maps (thresholded at $\mathrm{\lvert t \\rvert \geq 3}$) of the second-level analysis for block stimulation protocols, divided by implant category group.
                Slices are centered on VTA coordinates ($\mathrm{RAS = 0.5/-3.2/-4.5}$) and on the largest cluster, respectively.
                ',
        label='fig:var',
        )}
\end{sansmath}

\begin{sansmath}
\sessionpy{pytex_subfigs(
        [
                {'script':'scripts/implant_coordinates_block.py', 'label':'subjects', 'conf':'article/1col_2x3.conf', 'options_pre':'{.48\\textwidth}',
			'caption':'VTA activation intensity for block stimulation.',
                        'label':'fig:impl_block',
                        },
                {'script':'scripts/implant_coordinates_phasic.py', 'label':'tasks','conf':'article/1col_2x3.conf', 'options_pre':'{.48\\textwidth}',
			'caption':'VTA activation intensity for phasic stimulation.',
                        'label':'fig:impl_phasic',
                        },
                ],
        caption='
                \\textbf{Highest-sensitivity implant coordinates show different distribution for block and phasic stimulation.}
                Depicted are depth and posteroanterior coordinate-resolved distributions of implant targets and VTA activation intensities; dot diameters indicate sample size.
                ',
        label='fig:impl',
        )}
\end{sansmath}



\begin{sansmath}
\py[var]{pytex_subfigs(
        [
                {'script':'scripts/f_vta_subjects_t.py', 'label':'subjects', 'conf':'article/3x1.conf', 'options_pre':'{.35\\textwidth}',
                        },
                {'script':'scripts/implant_depth.py', 'label':'subjects', 'conf':'article/3x1.conf', 'options_pre':'{.32\\textwidth}',
                        },
                {'script':'scripts/implant_pa.py', 'label':'subjects', 'conf':'article/3x1.conf', 'options_pre':'{.32\\textwidth}',
                        },
                ],
        caption='
                Multivariate (depth and posteroantior) implant coordinate comparisons of signal intensity in the VTA region of interest.
                ',
        label='fig:var',
        )}
\end{sansmath}

Modelling interaction effects we observe the depth
(\sessionpy{boilerplate.anova(expression='Q("Task Category")*C(Q("Depth rel. skull [mm]")) + Q("Task Category")*C(Q("PA rel. Bregma [mm]"))',factor='C(Q("Depth rel. skull [mm]"))')})
or the PA coordinates
(\sessionpy{boilerplate.anova(expression='C(Q("Depth rel. skull [mm]")) + C(Q("PA rel. Bregma [mm]"))',factor='C(Q("PA rel. Bregma [mm]"))')}).

For block stimulation protocols we observe the depth
(\sessionpy{boilerplate.anova(expression='C(Q("Depth rel. skull [mm]")) + C(Q("PA rel. Bregma [mm]"))',factor='C(Q("Depth rel. skull [mm]"))', task_category='Block')})
or the PA coordinates
(\sessionpy{boilerplate.anova(expression='C(Q("Depth rel. skull [mm]")) + C(Q("PA rel. Bregma [mm]"))',factor='C(Q("PA rel. Bregma [mm]"))', task_category='Block')}).

\begin{sansmath}
\sessionpy{pytex_subfigs(
        [
                {'script':'scripts/map_block_filtered_seed.py', 'label':'subjects', 'conf':'article/2x1_map.conf', 'options_pre':'{.48\\textwidth}',
                        'caption':'Slices centered on VTA. \\vspace{-.2em}',
                        'label':'seed_map',
                        },
                {'script':'scripts/map_block_filtered_seed_auto.py', 'label':'subjects', 'conf':'article/2x1_map.conf', 'options_pre':'{.48\\textwidth}',
                        'caption':'Slices centered on langest cluster. \\vspace{-.2em}',
                        'label':'seed_map_auto',
                        },
                {'script':'scripts/distributions_block_filtered_seed.py', 'label':'tasks', 'conf':'article/distributions.conf', 'options_pre':'{.95\\textwidth}',
                        'caption':'Distribution densities in the 10 regions with the highest functional connectivity to the VTA seed.',
                        'label':'seed_dist',
                        },
                ],
        caption='
                \\textbf{Seed-based functional connectivity of VTA dopaminetgic projections shows laterality and a primary projection weighting towards the Nucleus Accumbens.}
                Depicted are volumetric population t-statistic maps --- \subref{fig:seed_map} and \subref{fig:seed_map_auto} --- thresholded at $\mathrm{t \geq 3}$, as well as a break-down of activation along atlas parcellation regions \subref{fig:seed_dist}.
                ',
        label='fig:seed',
        options_pre='\\centering',
        )}
\end{sansmath}

\section{Other Filtering Methods}

\begin{sansmath}
\sessionpy{pytex_subfigs(
        [
                {'script':'scripts/map_block_filtered_auto_.py', 'label':'subjects', 'conf':'article/3x1_map.conf', 'options_pre':'{.32\\textwidth}',
                        'caption':'Block stimulation of best implant category group.',
                        'label':'bs_f',
                        },
                {'script':'scripts/map_block_other_auto_.py', 'label':'subjects', 'conf':'article/3x1_map.conf', 'options_pre':'{.32\\textwidth}',
                        'caption':'Block stimulation of eliminated implant category group.',
                        'label':'bs_o',
                        },
                {'script':'scripts/implant_coordinates.py', 'label':'subjects', 'conf':'article/3x1_withmap.conf', 'options_pre':'{.32\\textwidth}',
                        'caption':'Implant group segmentation (dotted markers indicate best category).',
                        'label':'bs_c',
                        },
                ],
        caption='
                \\textbf{Implant category grouping of block stimulation trials (determined for all stimulation protocols) does not show better projection segmentation than block stimulation implant category grouping.}
                Depicted are t-statistic maps (thresholded at $\mathrm{t \geq 3}$) of the second-level analysis for block stimulation protocols, divided by implant category group with slices are centered on the largest cluster --- \subref{fig:bs_f} and \subref{fig:bs_o}.
                ',
        label='fig:bs_',
        )}
\end{sansmath}

\begin{sansmath}
\sessionpy{pytex_subfigs(
        [
                {'script':'scripts/map_block_filteredManual.py', 'label':'subjects', 'conf':'article/2x1_map.conf', 'options_pre':'{.48\\textwidth}',
                        'caption':'Block stimulation of rostralmost implant category group, slices centered on VTA. \\vspace{1.2em}'
                        },
                {'script':'scripts/map_block_filtered_autoManual.py', 'label':'subjects', 'conf':'article/2x1_map.conf', 'options_pre':'{.48\\textwidth}',
                        'caption':'Block stimulation of rostralmost implant category group, slices centered on langest cluster. \\vspace{1.2em}'
                        },
                {'script':'scripts/map_block_otherManual.py', 'label':'subjects', 'conf':'article/2x1_map.conf', 'options_pre':'{.48\\textwidth}',
                        'caption':'Block stimulation of caudalmost implant category group, slices centered on VTA.'
                        },
                {'script':'scripts/map_block_other_autoManual.py', 'label':'subjects', 'conf':'article/2x1_map.conf', 'options_pre':'{.48\\textwidth}',
                        'caption':'Block stimulation of caudalmost implant category group, slices centered on langest cluster.'
                        },
                ],
        caption='
                \\textbf{Implant category grouping of block stimulation trials (determined by PA coordinate separation at -3.25) does not show better projection segmentation than block stimulation implant category grouping.}
                Depicted are statistical maps (thresholded at $\mathrm{t \geq 3}$) of the second-level analysis for block stimulation protocols, divided by implant category group.
                Slices are centered on VTA coordinates ($\mathrm{RAS = 0.5/-3.2/-4.5}$) and on the largest cluster, respectively.
                ',
        label='fig:var',
        )}
\end{sansmath}

\begin{sansmath}
\sessionpy{pytex_subfigs(
        [
                {'script':'scripts/map_block.py', 'label':'subjects', 'conf':'article/2x1_map.conf', 'options_pre':'{.48\\textwidth}',
                        'caption':'Block stimulation, slices centered on VTA. Thresholded at \\vspace{1.2em}'
                        },
                {'script':'scripts/map_block_auto.py', 'label':'subjects', 'conf':'article/2x1_map.conf', 'options_pre':'{.48\\textwidth}',
                        'caption':'Block stimulation, slices centered on langest cluster. \\vspace{1.2em}'
                        },
                {'script':'scripts/map_phasic.py', 'label':'subjects', 'conf':'article/2x1_map.conf', 'options_pre':'{.48\\textwidth}',
                        'caption':'Phasic stimulation, slices centered on VTA.'
                        },
                {'script':'scripts/map_phasic_auto.py', 'label':'subjects', 'conf':'article/2x1_map.conf', 'options_pre':'{.48\\textwidth}',
                        'caption':'Phasic stimulation, slices centered on langest cluster.'
                        },
                ],
        caption='
                \\textbf{No negative activation patterns are salient upon either block or phasic VTA stimulation.}
                Depicted are t-statistic maps (thresholded at $\mathrm{\lvert t \\rvert \geq 3}$) of the second-level analysis divided by stimulation category and binning all implant coordinates.
                Slices are centered on VTA coordinates ($\mathrm{RAS = 0.5/-3.2/-4.5}$) and on the largest cluster, respectively.
                ',
        label='fig:all',
        )}
\end{sansmath}

\begin{sansmath}
\sessionpy{pytex_subfigs(
        [
                {'script':'scripts/map_block_filtered_seed_controlled.py', 'label':'subjects', 'conf':'article/2x1_map.conf', 'options_pre':'{.48\\textwidth}',
                        'caption':'Slices centered on VTA.'
                        },
                {'script':'scripts/map_block_filtered_seed_controlled_auto.py', 'label':'subjects', 'conf':'article/2x1_map.conf', 'options_pre':'{.48\\textwidth}',
                        'caption':'Slices centered on langest cluster.'
                        },
                ],
        caption='
                Depicted are t-statistic maps (thresholded at $\mathrm{t \geq 3}$) of the second-level analysis for block stimulation task VTA seed functional connectivity, observed in best-implant transgenic animals, as opposed to negative control animals.
                Slices are centered on VTA coordinates ($\mathrm{RAS = 0.5/-3.2/-4.5}$) and on the largest cluster, respectively.
                This comparison is only provided for the sake of completeness and analogy with the stimulus-evoked analysis.
                Conceptually this comparison is not of primary interest, since seed-based functional connectivity attempts to include precisely the baseline functioning of the system into the evaluation.
                ',
        label='fig:scc',
        )}
\end{sansmath}

\begin{sansmath}
\sessionpy{pytex_subfigs(
        [
                {'script':'scripts/map_block_filtered_controlled.py', 'label':'subjects', 'conf':'article/2x1_map.conf', 'options_pre':'{.48\\textwidth}',
                        'caption':'Slices centered on VTA. \\vspace{1.2em}'
                        },
                {'script':'scripts/map_block_filtered_controlled_auto.py', 'label':'subjects', 'conf':'article/2x1_map.conf', 'options_pre':'{.48\\textwidth}',
                        'caption':'Slices centered on langest cluster. \\vspace{1.2em}'
                        },
                {'script':'scripts/map_block_filtered_controlled_all.py', 'label':'subjects', 'conf':'article/2x1_map.conf', 'options_pre':'{.48\\textwidth}',
                        'caption':'Positive and negative values, slices centered on VTA.'
                        },
                {'script':'scripts/map_block_filtered_controlled_all_auto.py', 'label':'subjects', 'conf':'article/2x1_map.conf', 'options_pre':'{.48\\textwidth}',
                        'caption':'Positive and negative values, slices centered on langest cluster.'
                        },
                ],
        caption='
                Depicted are statistical maps (thresholded at $\mathrm{\lvert t \\rvert \geq 3}$) of the second-level analysis for the whole brain GLM during block stimulation of the best implant category group.
                Slices are centered on VTA coordinates ($\mathrm{RAS = 0.5/-3.2/-4.5}$) and on the largest cluster, respectively.
                ',
        label='fig:var',
        )}
\end{sansmath}
