\renewcommand{\thetable}{S\arabic{table}}
\setcounter{figure}{0}
\renewcommand{\thefigure}{S\arabic{figure}}

\section{Supplementary Materials}
\py{
        pytex_tab(inner_tabular=boilerplate.events_tab('data/CogB.tsv'),
                label='CogB',
                caption='“CogB” stimulation protocol.',
                options_pre='[h!] \\scriptsize \\centering \\resizebox{\\columnwidth}{!}{',
                data='data/CogB.tsv',
                options_post='}',
                )
        }
\py{
        pytex_tab(inner_tabular=boilerplate.events_tab('data/CogBr.tsv'),
                label='CogBr',
                caption='“CogBr” stimulation protocol.',
                options_pre='[h!] \\scriptsize \\centering \\resizebox{\\columnwidth}{!}{',
                data='data/CogBr.tsv',
                options_post='}',
                )
        }
\py{
        pytex_tab(inner_tabular=boilerplate.events_tab('data/CogBl.tsv'),
                label='CogBl',
                caption='“CogBl” stimulation protocol.',
                options_pre='[h!] \\scriptsize \\centering \\resizebox{\\columnwidth}{!}{',
                data='data/CogBl.tsv',
                options_post='}',
                )
        }
\py{
        pytex_tab(inner_tabular=boilerplate.events_tab('data/CogBm.tsv'),
                label='CogBm',
                caption='“CogBm” stimulation protocol.',
                options_pre='[h!] \\scriptsize \\centering \\resizebox{\\columnwidth}{!}{',
                data='data/CogBm.tsv',
                options_post='}',
                )
        }
\py{
        pytex_tab(inner_tabular=boilerplate.events_tab('data/CogMwf.tsv'),
                label='CogMwf',
                caption='“CogMwf” stimulation protocol.',
                options_pre='[h!] \\scriptsize \\centering \\resizebox{\\columnwidth}{!}{',
                data='data/CogMwf.tsv',
                options_post='}',
                )
        }
\py{
        pytex_tab(inner_tabular=boilerplate.events_tab('data/CogP.tsv'),
                label='CogP',
                caption='“CogP” stimulation protocol.',
                options_pre='[h!] \\scriptsize \\centering \\resizebox{\\columnwidth}{!}{',
                data='data/CogP.tsv',
                options_post='}',
                )
        }
\py{
        pytex_tab(inner_tabular=boilerplate.events_tab('data/JPogP.tsv'),
                label='JPogP',
                caption='“JPogP” stimulation protocol.',
                options_pre='[h!] \\scriptsize \\centering \\resizebox{\\columnwidth}{!}{',
                data='data/JPogP.tsv',
                options_post='}',
                )
        }


\begin{sansmath}
\py{pytex_subfigs(
        [
                {'script':'scripts/map_filtered_all.py', 'label':'subjects', 'conf':'article/1col_map.conf', 'options_pre':'{.48\\textwidth}',
                        'caption':'Best coordinate group with slices centered on VTA. \\vspace{1.2em}'
                        },
                {'script':'scripts/map_filtered_all_auto.py', 'label':'subjects', 'conf':'article/1col_map.conf', 'options_pre':'{.48\\textwidth}',
                        'caption':'Best coordinate group centered on langest cluster. \\vspace{1.2em}'
                        },
                {'script':'scripts/map_other_all.py', 'label':'subjects', 'conf':'article/1col_map.conf', 'options_pre':'{.48\\textwidth}',
			'caption':'Remaining coordinate group centered on VTA.'
                        },
                {'script':'scripts/map_other_all_auto.py', 'label':'subjects', 'conf':'article/1col_map.conf', 'options_pre':'{.48\\textwidth}',
			'caption':'Remaining coordinate group centered on largest cluster.'
                        },
                ],
        caption='
                \\textbf{Best coordinate group scans elicit activity in the Striatum and the Nucleus Accumbens, whereas remaining scans do not.}
                Depicted are statistical maps (thresholded at $\mathrm{\lvert t \\rvert \geq 3}$) of the second-level analysis for block stimulation protocols, comparing different subject groups segmented by implant coordinates --- best coordinate group ($\mathrm{PA \geq -3.3 ; IS \geq -4.4}$) and remaining scans.
                Slices are centered on VTA coordinates ($\mathrm{RAS = 0.5/-3.2/-4.5}$) and on the largest cluster, respectively.
                ',
        label='fig:var',
        )}
\end{sansmath}

\begin{sansmath}
\py{pytex_subfigs(
        [
                {'script':'scripts/implant_depth.py', 'label':'subjects', 'conf':'article/1col.conf', 'options_pre':'{.48\\textwidth}',
                        'caption':'Implant coordinate comparison for block stimulation scans only, sliced by depth. \\vspace{1.2em}'
                        ,},
                {'script':'scripts/implant_pa.py', 'label':'subjects', 'conf':'article/1col.conf', 'options_pre':'{.48\\textwidth}',
			'caption':'Implant coordinate comparison for block stimulation scans only, sliced by PA coordinates. \\vspace{1.2em}'
                        ,},
                {'script':'scripts/implant_depth_block.py', 'label':'subjects', 'conf':'article/1col.conf', 'options_pre':'{.48\\textwidth}',
			'caption':'Implant coordinate comparison for block stimulation scans only, sliced by depth.'
                        ,},
                {'script':'scripts/implant_pa_block.py', 'label':'subjects', 'conf':'article/1col.conf', 'options_pre':'{.48\\textwidth}',
			'caption':'Implant coordinate comparison for block stimulation scans only, sliced by PA coordinates.'
                        ,},
                ],
        caption='
                Multivariate (subject and stimulation protocol) comparisons of significance and signal intensity at the whole-brain level or restricted to the VTA region of interest.
                ',
        label='fig:var',
        )}
\end{sansmath}

Modelling interaction effects we observe the depth
(\py{boilerplate.anova(expression='Q("Task Category")*C(Q("Depth rel. skull [mm]")) + Q("Task Category")*C(Q("PA rel. Bregma [mm]"))',factor='C(Q("Depth rel. skull [mm]"))')})
or the PA coordinates
(\py{boilerplate.anova(expression='C(Q("Depth rel. skull [mm]")) + C(Q("PA rel. Bregma [mm]"))',factor='C(Q("PA rel. Bregma [mm]"))')}).

For block stimulation protocols we observe the depth
(\py{boilerplate.anova_block(expression='C(Q("Depth rel. skull [mm]")) + C(Q("PA rel. Bregma [mm]"))',factor='C(Q("Depth rel. skull [mm]"))')})
or the PA coordinates
(\py{boilerplate.anova_block(expression='C(Q("Depth rel. skull [mm]")) + C(Q("PA rel. Bregma [mm]"))',factor='C(Q("PA rel. Bregma [mm]"))')}).

\begin{sansmath}
\py{pytex_subfigs(
        [
                {'script':'scripts/f_vta_subjects_t.py', 'label':'subjects', 'conf':'article/1col_2x3.conf', 'options_pre':'{.48\\textwidth}',
			'caption':'VTA signal intensity across subjects.'
                        ,},
                {'script':'scripts/f_vta_tasks_t.py', 'label':'tasks','conf':'article/1col_2x3.conf', 'options_pre':'{.48\\textwidth}',
                        'caption':'VTA signal intensity across stimulation protocols.'
                        ,},
                ],
        caption='
                Multivariate (subject and stimulation protocol) comparisons of significance and signal intensity at the whole-brain level or restricted to the VTA region of interest.
                ',
        label='fig:var',
        )}
\end{sansmath}

\begin{sansmath}
\py{pytex_subfigs(
        [
                {'script':'scripts/implant_coordinates_block.py', 'label':'subjects', 'conf':'article/1col_2x3.conf', 'options_pre':'{.48\\textwidth}',
			'caption':'VTA signal intensity across subjects.'
                        ,},
                {'script':'scripts/implant_coordinates_phasic.py', 'label':'tasks','conf':'article/1col_2x3.conf', 'options_pre':'{.48\\textwidth}',
                        'caption':'VTA signal intensity across stimulation protocols.'
                        ,},
                ],
        caption='
                Multivariate (subject and stimulation protocol) comparisons of significance and signal intensity at the whole-brain level or restricted to the VTA region of interest.
                ',
        label='fig:var',
        )}
\end{sansmath}

