\renewcommand{\thetable}{S\arabic{table}}
\setcounter{figure}{0}
\renewcommand{\thefigure}{S\arabic{figure}}

\section{Supplementary Materials}
\sessionpy{
        pytex_tab(inner_tabular=boilerplate.events_tab('data/CogB.tsv'),
                label='CogB',
                caption='Block stimulation protocol, coded “CogB”.',
                options_pre='[h!] \\scriptsize \\centering \\resizebox{\\columnwidth}{!}{',
                data='data/CogB.tsv',
                options_post='}',
                )
        }
\sessionpy{
        pytex_tab(inner_tabular=boilerplate.events_tab('data/CogBr.tsv'),
                label='CogBr',
                caption='Block stimulation protocol, coded “CogBr”.',
                options_pre='[h!] \\scriptsize \\centering \\resizebox{\\columnwidth}{!}{',
                data='data/CogBr.tsv',
                options_post='}',
                )
        }
\sessionpy{
        pytex_tab(inner_tabular=boilerplate.events_tab('data/CogBl.tsv'),
                label='CogBl',
                caption='Block stimulation protocol, coded “CogBl”.',
                options_pre='[h!] \\scriptsize \\centering \\resizebox{\\columnwidth}{!}{',
                data='data/CogBl.tsv',
                options_post='}',
                )
        }
\sessionpy{
        pytex_tab(inner_tabular=boilerplate.events_tab('data/CogBm.tsv'),
                label='CogBm',
                caption='Block stimulation protocol, coded “CogBm”.',
                options_pre='[h!] \\scriptsize \\centering \\resizebox{\\columnwidth}{!}{',
                data='data/CogBm.tsv',
                options_post='}',
                )
        }
\sessionpy{
        pytex_tab(inner_tabular=boilerplate.events_tab('data/CogMwf.tsv'),
                label='CogMwf',
                caption='Block stimulation protocol, coded “CogMwf”.',
                options_pre='[h!] \\scriptsize \\centering \\resizebox{\\columnwidth}{!}{',
                data='data/CogMwf.tsv',
                options_post='}',
                )
        }
\sessionpy{
        pytex_tab(inner_tabular=boilerplate.events_tab('data/CogP.tsv'),
                label='CogP',
                caption='Phasic stimulation protocol, coded “CogP”.',
                options_pre='[h!] \\scriptsize \\centering \\resizebox{\\columnwidth}{!}{',
                data='data/CogP.tsv',
                options_post='}',
                )
        }
\sessionpy{
        pytex_tab(inner_tabular=boilerplate.events_tab('data/JPogP.tsv'),
                label='JPogP',
                caption='Phasic stimulation protocol, coded “JPogP”.',
                options_pre='[h!] \\scriptsize \\centering \\resizebox{\\columnwidth}{!}{',
                data='data/JPogP.tsv',
                options_post='}',
                )
        }

\clearpage

In a linear modelling of the implant coordinate variables, the VTA mean t statistic is found sensitive only to
the stimulation protocol category
(\sessionpy{boilerplate.anova(
        expression='Q("Depth rel. skull [mm]")*Q("PA rel. Bregma [mm]") + Q("Task Category")',
        factor='Q("Task Category")',
        )}),
but not the stimulation target depth
(\sessionpy{boilerplate.anova(
        expression='Q("Depth rel. skull [mm]")*Q("PA rel. Bregma [mm]") + Q("Task Category")',
        factor='Q("Depth rel. skull [mm]")',
        )}),
the stimulation target posteroanterior (PA) coordinates
(\sessionpy{boilerplate.anova(
        expression='Q("Depth rel. skull [mm]")*Q("PA rel. Bregma [mm]") + Q("Task Category")',
        factor='Q("PA rel. Bregma [mm]")',
        )}),
and the interaction of the depth and PA target coordinates
(\sessionpy{boilerplate.anova(
        expression='Q("Depth rel. skull [mm]")*Q("PA rel. Bregma [mm]") + Q("Task Category")',
        factor='Q("Depth rel. skull [mm]"):Q("PA rel. Bregma [mm]")',
        )}).



\begin{sansmath}
\py[var]{pytex_subfigs(
        [
                {'script':'scripts/f_vta_subjects_t.py', 'label':'mvss', 'conf':'article/3x1_labelspace.conf', 'options_pre':'{.32\\textwidth}',
                        },
                {'script':'scripts/implant_depth.py', 'label':'mvsd', 'conf':'article/3x1.conf', 'options_pre':'{.32\\textwidth}',
                        },
                {'script':'scripts/implant_pa.py', 'label':'mvsp', 'conf':'article/3x1.conf', 'options_pre':'{.32\\textwidth}',
                        },
                ],
        caption='
                Multifactorial (depth and posteroantior) implant coordinate comparisons of signal intensity in the VTA region of interest.
                Protocols coded as in \cref{tab:CogB,tab:CogBr,tab:CogBl,tab:CogBm,tab:CogMwf,tab:CogP,tab:JPogP}.
                ',
        label='fig:mvs',
        )}
\end{sansmath}

\begin{sansmath}
\sessionpy{pytex_subfigs(
        [
                {'script':'scripts/map_block_filtered.py', 'conf':'article/2x1_map.conf', 'options_pre':'{.48\\textwidth}',
                        'caption':'Block stimulation, slices centered on largest cluster. \\vspace{1.2em}',
                        'label':'blocka',
                        },
                {'script':'scripts/map_phasic_filtered_auto.py', 'conf':'article/2x1_map.conf', 'options_pre':'{.48\\textwidth}',
                        'caption':'Phasic stimulation, slices centered on largest cluster.',
                        'label':'phasica',
                        },
                ],
        caption='
                \\textbf{No negative activation patterns are salient upon block VTA stimulation, and no coherent activation patterns of any sort after phasic VTA stimulation.}
                Depicted are t-statistic maps (thresholded at $\mathrm{\\lvert t \\rvert \geq 3}$) of second-level analyses, divided by stimulation category and binning all implant coordinates.
                Slices are centered on the VTA coordinates ($\mathrm{RAS = 0.5/-3.2/-4.5}$) and the largest cluster, respectively.
                All maps are adjusted for the wild type control stimulation effects.
                ',
        label='fig:all',
        )}
\end{sansmath}

\begin{sansmath}
\sessionpy{pytex_subfigs(
        [
                {'script':'scripts/map_block_filtered_controlled_auto.py', 'conf':'article/3x1_map.conf', 'options_pre':'{.32\\textwidth}',
                        'options_post':'\\vspace{1em}',
                        'caption':'Block stimulation of best implant category group (according to VTA activation), centered on largest cluster.',
                        'label':'ocf',
                        },
                {'script':'scripts/map_block_other_controlled.py', 'conf':'article/3x1_map.conf', 'options_pre':'{.32\\textwidth}',
                        'options_post':'\\vspace{1em}',
                        'caption':'Block stimulation of rejected implant category group (according to VTA activation), centered on VTA.',
                        'label':'oco',
                        },
                {'script':'scripts/implant_coordinates_block_mini.py', 'conf':'article/3x1_withmap.conf', 'options_pre':'{.32\\textwidth}',
                        'options_post':'\\vspace{1em}',
                        'caption':'VTA activation-based implant coordinate classification (dotted markers indicate best category).',
                        'label':'occ',
                        },
                {'script':'scripts/map_block_filtered_controlledManual.py', 'conf':'article/3x1_map.conf', 'options_pre':'{.32\\textwidth}',
                        'caption':'Block stimulation of rostralmost implant category group, centered on VTA.',
                        'label':'pcf',
                        },
                {'script':'scripts/map_block_other_controlledManual.py', 'conf':'article/3x1_map.conf', 'options_pre':'{.32\\textwidth}',
                        'caption':'Block stimulation of caudalmost implant category group, centered on largest cluster.',
                        'label':'pco',
                        },
                {'script':'scripts/implant_coordinates_pa_mini.py', 'conf':'article/3x1_withmap.conf', 'options_pre':'{.32\\textwidth}',
                        'caption':'PA-coordinate implant classification (dotted markers indicate rostralmost category).',
                        'label':'pcc',
                        },
                ],
        caption='
                \\textbf{PA-coordinate-based classification does not show a better projection segmentation than block trial-based classification.}
                Depicted are t-statistic maps (centerd on largest cluster, thresholded at $\mathrm{t \geq 3}$) of the second-level analysis for block stimulation protocols, divided into best and rejected \\textbf{(\subref{fig:ocf}, \subref{fig:oco})}, or rostralmost and caudalmost \\textbf{(\subref{fig:pcf}, \subref{fig:pco})}.
                All maps are adjusted for the wild type control stimulation effects.
                ',
        label='fig:oc',
        )}
\end{sansmath}

\begin{sansmath}
\sessionpy{pytex_subfigs(
        [
                {'script':'scripts/map_block_filtered.py', 'label':'subjects', 'conf':'article/2x1_map.conf', 'options_pre':'{.48\\textwidth}',
                        'caption':'Block stimulation of best implant category group',
                        },
                {'script':'scripts/map_block_other.py', 'label':'subjects', 'conf':'article/2x1_map.conf', 'options_pre':'{.48\\textwidth}',
                        'caption':'Block stimulation of rejected implant category group.',
                        },
                ],
        caption='
                \\textbf{The uncorrected population-level response to block stimulation does not qualitatively differ from the wild type control corrected results in \cref{fig:mbfc,fig:mboc}.}
                Depicted are wildtype-control-uncorrected t-statistic maps (thresholded at $\mathrm{t \geq 3}$) of the second-level analysis for block stimulation protocols, divided by implant category group.
                Slices are centered on the VTA region of interest.
                ',
        label='fig:uncontrolled',
        )}
\end{sansmath}

\begin{sansmath}
\sessionpy{pytex_subfigs(
        [
                {'script':'scripts/map_block_control.py', 'label':'subjects', 'conf':'article/2x1_map.conf', 'options_pre':'{.48\\textwidth}',
                        'caption':'Slices centered on VTA. \\vspace{-.2em}',
                        'label':'control_map',
                        },
                {'script':'scripts/map_block_control_auto.py', 'label':'subjects', 'conf':'article/2x1_map.conf', 'options_pre':'{.48\\textwidth}',
                        'caption':'Slices centered on largest cluster. \\vspace{-.2em}',
                        'label':'control_mapa',
                        },
                {'script':'scripts/distributions_block_control.py', 'label':'tasks', 'conf':'article/distributions.conf', 'options_pre':'{.95\\textwidth}',
                        'caption':'Distribution densities of t-statistic values in the 10 most strongly activated areas.',
                        'label':'control_dist',
                        },
                ],
        caption='
                \\textbf{Block stimulation in wild type control animals produces no large activation clusters, yet scattered activation hints at some visual excitation and heating artefacts.}
                Depicted are volumetric population t-statistic maps \\textbf{(\subref{fig:control_map}, \subref{fig:control_mapa})} --- thresholded at $\mathrm{t \geq 3}$, as well as a break-down of activation along atlas parcellation regions \\textbf{(\subref{fig:control_dist})}.
                ',
        label='fig:control',
        options_pre='\\centering',
        )}
\end{sansmath}

\begin{sansmath}
\sessionpy{pytex_subfigs(
        [
                {'script':'scripts/map_block_filtered_seed_controlled.py', 'label':'subjects', 'conf':'article/2x1_map.conf', 'options_pre':'{.48\\textwidth}',
                        'caption':'Slices centered on VTA.'
                        },
                {'script':'scripts/map_block_filtered_seed_controlled_auto.py', 'label':'subjects', 'conf':'article/2x1_map.conf', 'options_pre':'{.48\\textwidth}',
                        'caption':'Slices centered on largest cluster.'
                        },
                ],
        caption='
                Depicted are t-statistic maps (thresholded at $\mathrm{t \geq 3}$) of the second-level analysis for block stimulation task VTA seed functional connectivity, observed in the best implant category, corrected for the negative control baseline.
                Slices are centered on the VTA coordinates ($\mathrm{RAS = 0.5/-3.2/-4.5}$) and the largest cluster, respectively.
                This comparison is only provided for the sake of completeness and analogy with the stimulus-evoked analysis.
                Conceptually this comparison is not of primary interest, since seed-based functional connectivity attempts to include precisely the baseline functioning of the system into the evaluation.
                ',
        label='fig:sc',
        )}
\end{sansmath}

%\begin{sansmath}
%\sessionpy{pytex_subfigs(
%        [
%                {'script':'scripts/map_features_filtered.py', 'conf':'article/2asymetric_map.conf',
%                        'options_pre':'{.39\\textwidth}',
%                        'options_pre_caption':'\\vspace{-1.6em}',
%                        'caption':'Contrast between VTA functional activation and structural projections, showing slices centered on largest cluster, statistical map thresholded at $\mathrm{3 \geq t \geq 3}$. \\vspace{.8em}',
%                        'label':'mff',
%                        },
%                {'script':'scripts/distributions_features_filtered_neg.py', 'conf':'article/2asymetric_distributions.conf',
%                        'options_pre':'{.595\\textwidth}\\vspace{-1.6em}',
%                        'options_pre_caption':'\\vspace{-1.5em}',
%                        'caption':'Distribution densities of t-statistics, showing the regions where VTA structural projections exceed functional actiivation most strongly.\\vspace{.8em}',
%                        'label':'dffn',
%                        },
%                ],
%        caption='
%                \\textbf{Comparing VTA functional activation to structural projection data reveals good correspondence, with deviations involving the contralateral and the dorsal ipsilateral striatum.}
%                Depicted are contrast statistic values, taking into account variability across subjects \\textbf{(\subref{fig:mffa}, \subref{fig:dff})}, alongside an overlay \\textbf{(\subref{fig:ffc})} and correlation analyses \\textbf{(\subref{fig:fcr}, \subref{fig:fcv})} of the the population-level functional and structural statistic maps.
%                ',
%        label='fig:dummy',
%        options_pre='\\centering\n\\vspace{-2em}',
%        )}
%\end{sansmath}

\begin{sansmath}
\sessionpy{pytex_fig(
        'scripts/features_filtered_coronal_contour.py',
        conf='article/coronal.conf',
        caption='
                Coronal slice overlay, showing the VTA functional activation t-statistic heatmap (as in \\textbf{\\cref{fig:mbfc}}), and the VTA structural projection outline, both thresholded at $\mathrm{t \geq 3}$.
                Interpretation of this figure as showcasing a complementarity in the patterns is cautioned, as qualitative inspection of thresholded data does not accurately capture variation in the statistic distributions.
                For statements regarding the compariosn of functional activation and structural projection, \cref{fig:fcr,fig:fcv,fig:ffc} are more suitable. 
                \\vspace{-1em}
                ',
        label='ffcc',
        options_pre='\\centering\n',
        environment='figure*',
        figure_format='pdf',
        )}
\end{sansmath}

