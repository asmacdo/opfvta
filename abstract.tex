Ascending dopaminergic projections from neurons located in the Ventral Tegmental Area (VTA) are key to the etiology, dysfunction, and control of motivation, learning, and addiction.
Due to evolutionary conservation of this nucleus and the extensive use of mice as disease models, establishing an assay for VTA dopaminergic signalling in the mouse brain is crucial for the translational investigation of neuronal function phenotypes of diseases and interventions.
In this article we use optogenetic stimulation for targeted VTA dopaminergic neuron control, in combination with functional Magnetic Resonance Imaging (fMRI), a method widely used in human deep brain imaging.
We present the first whole-brain opto-fMRI map of VTA dopaminergic activation in the mouse, and show that dopaminergic system function is consistent with but diverges in a few key aspects from its structure.
While the activation map predominantly includes and excludes target areas according to their relative projection densities (e.g. strong activation of the nucleus accumbens and low activation of the hippocampus), it also includes areas for which a structural connection is not well established (such as the dorsal striatum).
We further detail assay variability with regard to multiple experimental parameters, including stimulation protocol and implant position, and provide detailed evidence-based recommendations for assay reuse.
