Ascending dopaminergic projections from neurons located in the Ventral Tegmental Area (VTA) are key to the etiology, dysfunction, and control of motivation, learning, and addiction.
Due to the evolutionary conservation of this nucleus and the extensive use of mice as disease models, establishing an assay for VTA dopaminergic activation in the mouse is highly relevant for the translational investigation of neuronal function phenotypes of diseases and interventions.
In this article we use optogenetic stimulation for targeted VTA dopaminergic neuron control, and acquire data via functional Magnetic Resonance Imaging (fMRI), a method widely used in human deep brain imaging.
We present the first whole-brain opto-fMRI map of VTA dopaminergic activation in the mouse, and show that dopaminergic system function is coherent with but diverges in a few key aspects from its structure.
This activation map predominantly includes structurally connected target areas (e.g. the nucleus accumbens), excludes structurally connected areas with low projection density (e.g. the hippocampus), and includes areas for which a structural connection is not well established (e.g. the dorsal striatum).
We further detail assay variability with regard to multiple experimental parameters, including stimulation protocol and implant position --- and offer detailed evidence-based recommendations for assay re-use.
