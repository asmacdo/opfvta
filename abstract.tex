Ascending dopaminergic projections from neurons located in the Ventral Tegmental Area (VTA) are key to the etiology, dysfunction, and control of motivation, learning, and addiction.
Due to evolutionary conservation of this nucleus and the extensive use of mice as disease models, establishing an assay for VTA dopaminergic signalling in the mouse brain is crucial for the translational investigation of neuronal function phenotypes of diseases and interventions.
In this article we use optogenetic stimulation directed at VTA dopaminergic neurons in combination with functional Magnetic Resonance Imaging (fMRI), a method widely used in  human deep brain imaging.
We present a comprehensive assay producing the first whole-brain opto-fMRI map of dopaminergic activation in the mouse, and show that VTA dopaminergic system function is consistent with its structure, yet diverges in a few key aspects.
While the activation map predominantly highlights target areas according to their relative projection densities (e.g. strong activation of the nucleus accumbens and low activation of the hippocampus), it also includes areas for which a structural connection is not well established (such as the dorsomedial striatum).
We further detail the variability of the assay with regard to multiple experimental parameters, including stimulation protocol and implant position, and provide evidence-based recommendations for assay reuse, publishing both reference results and a reference analysis workflow implementation.
