Ascending dopaminergic projections rooted in the Ventral Tegmental Area (VTA) are involved in numerous phenomena of great neuropsychological interest, including motivation, learning, and addiction.
The study of dopaminergic signalling in humans is chiefly conducted via functional magnetic resonance imaging (fMRI), but is severely restricted in terms of molecular and cell biological interventions, as well as in terms of direct dopaminergic system control.
Optogenetic fMRI (opto-fMRI) in the mouse affords the possibility of direct dopaminergic control, alongside whole-brain imaging via the same modality used in human research.
As the dopaminergic system is evolutionarily well-conserved, this assay allows the identification of novel ways to modulate and categorize dopaminergic activity in both model animals and humans.
In this article we present a whole-brain fMRI map of baseline VTA dopaminergic transmission in the mouse.
Additionally, we explore the fundamentals of the assay and offer a comprehensive guide to best practices based on variation in multiple experimental parameters including implant positioning and stimulation protocols.
