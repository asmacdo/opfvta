Ascending dopaminergic projections rooted in the Ventral Tegmental Area (VTA) are involved in numerous phenomena of great neuropsychological interest, including motivation, learning, and addiction.
The study of dopaminergic signalling in humans is chiefly conducted via functional magnetic resonance imaging (fMRI), but is severely restricted in terms of molecular and cell biological interventions, as well as in terms of direct dopaminergic system control.
Optogenetic fMRI (opto-fMRI) in the mouse model organism, affords the possibility of direct dopaminergic control, as well as whole-brain imaging using a modality identical to that used in human studies.
As the dopaminergic system is also evolutionarily well-conserved, Work based upon this assay thus allows the identification of novel ways to modulate and categorize dopaminergic activity in both model animals and humans.
In this article we explore the fundamentals of the assay and offer a comprehensive guide to best practices based on variation in multiple experimental parameters including implant positioning and stimulation protocols.
