Ascending dopaminergic projections rooted in the Ventral Tegmental Area (VTA) are key to the etiology and control of motivation, learning, and addiction.
Due to VTA evolutionary conservation and the extensive use of mice as disease models, establishing a fingerprint of dopaminergic VTA function in the mouse is highly relevant both for translational application and as a tool to investigate interventions.
In this article we use optogenetic stimulation for targeted VTA dopaminergic control, and acquire data via the same modality used in human deep brain imaging.
We present a whole-brain opto-fMRI map of baseline VTA dopaminergic activation in the mouse, and show that dopaminergic system function diverges in meaningful ways from its structure.
This map includes prominent structurally connected areas (e.g. the nucleus accumbens), but also excludes structurally connected areas (e.g. the hippocampus), and includes areas for which a structural connection is not established (e.g. the dorsal striatum).
We further detail variability of the assay with regard to multiple experimental parameters, including stimulation protocol and implant position --- and offer detailed evidence-based recommendations for assay re-use.
